\subsection{Social Force models}
%	There should be an introduction to this section.. were going to compare the 		%
%	differnt ways of simulating crowds.. but wait then it's not only relevant for 		%
%	social force models. This might fit better into the Crowds section if its rewitten	%
%	correctly.																			%

%	Some history about the social force model might serve well as an introduction here	%
In this section we will first give a short presentation on Social Force models and their history,
and then compare the features of Social Force models with the features of other models that describes
the behavior of crowds.
\\

The idea behind social force models is that each individual, pedestrian $\alpha$, in the crowd acts 
on her own from a personal aim or goal and by behavioral reactions to the environment. 
Different forces such as other pedestrians, walls and obstacles are taken into account 
when pedestrian $\alpha$ walks toward her aim or goal, so that she does not walk 
into another pedestrian or wall. The forces influencing $\alpha$'s motion 
of direction are not forces acting on $\alpha$'s body, but has to be seen as a quantity 
that describes the motivation to act. Since $\alpha$ is used to the situations 
she is normally confronted with, her reactions will be rather automatic and 
determined by her experience of which reaction will be that best. Since the reaction 
is rather automatic it is possible to put the rules of pedestrian behavior into an equation 
of motion.\cite{social-force}

\subsection{Comparison with other crowd models}
There are other ways to model crowds than social force models such as Rule Based models, 
Cellular Automata models and Hybrid models. Below we will make a short explanation of each of the 
other crowd models, and then compare the features of the models.\\												%

\textbf{Rule Based models}\\
Rule Based models describe the movement of pedestrians through sets of basic rules. Each 
pedestrian apply collision detection and avoidance to prevent collision with other pedestrians. 
However they do not in general apply collision response, and therefore collisions and overlappping 
of the pedestrians may occur under some circumstances. Some models have applied stopping 
rules to prevent these situations.\cite{Comparison}

\textbf{Cellular Autamata models}\\
These models are discrete, deterministic and is made up of cells like the squares in a chessboard.
An artificial intelligence approach is used in Celluar Automata (CA) defined as mathematical idealisation of physical systems in which time and space are
discrete, and physical quantities take a finite set of discrete values. Pedestrian can not collide since the floor is dicretised and the pedestrians can
only move to free adjecent cells.\cite{Comparison}

\textbf{Hybrid models}\\
The Hybrid models has been made from social force models and rule based models. 
The motion of the pedestrians is caried out like the social force models, but is also based 
on the psychological and geometrical rules. It peforms collision detection and response and rules 
are applied depending on the pedestrian's personality and the state of the environment. 
\cite{Comparison}
\\

\subsubsection{Geometrical features of the models}
Each of the different models have some geomatrical features relatively important and 
unimportant according to our simulation.

\textbf{Shaking} Wether pedestrians appear to shake when simulating the model. When 
simulating the social force models the pedestrians seems to shake when cloggings and 
queues arise. This is caused by the modification of each pedestrians postition in each 
timestep. The Hybrid, CA and Rule Based model do not appear to shake while the simulation run.

\textbf{Discrete and Continuous movement} Wether the space and time are discretised. In the CA model the pedestrians move between discritised adjecent
cells in one time step, and therefore have limited direction to go. The other models do not discretise space and therefore allow the pedestrians to move
within continuous space.
 
\textbf{Overlapping} Wether overlapping of the pedestrians is possible. The rule based models and the CA models do have collision detection but not
all have collision responce. The CA models have rules so that a pedestrian can not walk into an occupied cell, but if two pedestrians want to pass each
other they step into the cell diagonal to their own cell, and in their way cross the other pedestrian in the intersection between the four cells.
Newer models of the rule based and CA have made a stopping rule so that overlapping can not accour. The Hybrid and social force models do collision responce
to minize the risk of overlapping. \cite{Comparison}

\textbf{Pushing} Wether pedestrians can have physical contact. If the pedestrians can have physical contact they will be able to push each other in some
direction. This ability is possible when using the Hybrid and the social force model. When simulating evacuations and chaotic events it has been observed
that people do have physical contact and push each other \cite{self-org}.

\textbf{Communication} Wether the pedestrians can share information about he environment. The HiDAC and som newer rule based models allows the pedestrians
to share information about the environment and to give orders. This feature is not included in social force models and CA. \cite{Comparison}

To illustrate different features of the different models the table below sums up the above mentioned.
\begin{center}
\begin{tabular}{lllll}
 & Social Forces & Rule Based & CA & Hybrid\\
Shaking avoidance     & - & + & + & +\\
Continuous space      & + & + & - & +\\
Overlapping avoidance & + & * & - & +\\
Pushing               & + & - & - & +\\
Communication         & - & * & - & +
\end{tabular}
\end{center}
Here ``+`` indicates that the feature is possible, ``-`` indicates it is not, and ''*'' that the model has been adjusted such that the feature have
become possible. \cite{Comparison}

The features we are interested in and find relevant in our simulation are continuous space, pushing and overlapping avoidance. The continuous space is
relevant for our simulaion to be as realistic as possible since we do not want our pedestrians to have at most nine directions to go each timestep.
The feature of pushing is also important to include since in panic sitiation it has been seen that people push each. This feature will then make our
simulation more realistic.
The overlapping avoidance we find relevant since this enables pedestrians to walk through each other.

%	Finishing off with a conclusion would be neat					%