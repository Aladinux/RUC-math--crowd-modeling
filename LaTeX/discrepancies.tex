\section{Discrepancies between results}
\label{sec:discrepancies}
In this section we discuss the different features of the model, problems when 
computing the model, limits of the model and ways of improving the model that 
we thought of when we went into more details of the model and started to 
simulate it.

There are several limitations the model that we encountered when we started 
analyse and simulate the model. In this section we discuss some relevant 
limitations of the model that we thought of.

The flexibility of the model in the sence that which situations it could model 
and some limitations of the agents and their awareness of the environment.
E. g. (Comparison article) some people fall in panic situations and become 
obstacles for other people, which is not a feature this model deals with. In 
our model none of the agents fall and become an obstacle force other agents as 
would be seen in real life panic situations.

Another flaw in the model is that the agents do not have way finding, which 
mean that they can not escape a more complex environment with a lot of rooms 
and corridors. As it is now we can only simulate simple cases as the squared 
room and the corridor.  An improvement of the model would then be to implement 
way finding for the agents, so that they can manoeuvre around in other more 
complex cases then the ones we have.

In other models as the HiDAC, agents can share information about the 
environment, such that the agents have a better idea of where the exit is/are 
(Comparison article). In a real panic situation you would expect people to 
communicate when evacuation a building.

Other models take into account herding behavior between the agents  
\cite{helbing00}, so that some people have tendency to follow other agents, 
while other agents have a tendency to go their own way,  and by doing so 
explore the environment for possible exits.

As the model is now there is no friction between the agents, so that agents 
that have higher velocity than the ones standing in their path, can slide in 
between slower moving agents in front of them, without them slowing down when 
they slide by. In real case scenario people would not slip by that easily 
since the friction between the people would slow them down.  Without friction 
between agents there would be less clogging in front of doors, and the time it 
takes the agents to exit a room would be less than the time expected. Thus the 
validation of the would not be as good as if there would be friction between 
the agents.

At the moment we can not simulate a case where the visibility is very low. The 
agents have a way point to follow, but if the visibility only is 1 meter, e. 
g. in a smoke filled room, the  agents would not have way point to follow, and 
hence round around with out purpose. Here the tangential forces would also 
help the agent to find the exit, but navigation through the tangential force 
from the walls.
