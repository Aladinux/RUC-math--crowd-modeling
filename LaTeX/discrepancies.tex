\section{Discrepancies between results}
\label{sec:discrepancies}

\subsection{Tangential forces}
Here we discuss the idea of improving the model by improving tangential forces, that is, the force parallel to the surface of an object.
The tangential forces can be used as collision avoidance, so that agents steer around obstacles or other agents \cite{tang}.
This would be relevant when simulating an environment where, e. g., and is standing in the middle of a symmetrical room and the agent's waypoint
is behind a big square pillar, with equal distance either way around, the agent would get stuck in front of the pillar without the tangential forces,
since the forces acting on the agent's motivation is cancelling each other out. 
In our case where two groups of agents are crossing each other in a corridor, the tangential force would make the agents to into account
the position of the agents in front of them and steer around them.

The tangential forces can also improve the model so it can simulate agents escaping a smoke filled room where we determine the visibility.
In this case of simulation the agents would walk randomly around untill they find a wall, and follow the wall around due to the tangential force.
This would be a way of implementing way finding to the model.

Friction between agents would also be possible to add to the model, since the tangential forces from other agents would make it
harder for agents to walk through a crowd. \cite{self-org} mention that friction is causing clogging in front of exits, since the people
get stuck in each other. In our simulation the agents are not clogging as heavily as \cite{self-org} mention it can be.
The time it take for the agent to leave a room in unrealistic low, and we think that implementing the friction by tangential forces
would make the simulation time more realistic.


