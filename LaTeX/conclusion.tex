% vim:ft=tex
\section{Conclusion}
\label{sec:conclusion}

\subsection{Further work}
Since the simulation we present in the report did't reproduce all the phenomena described in
section \ref{subsec:ThePhenomena}, if given more time our group would like to solve the discrepancies and add the possible solutions that has been talked about in section~\ref{sec:discussion}. The first thing we would implement into the model would be the frictional force, and make the frictional and repulsive forces velocity dependent. This should possibly give rise to the slow-is-faster effect and the freezing-by-heating effect, further more is should make the predictions made by the model more precise.
One could add path finding, enabling the model to simulate complex environments.

It would be interesting to examen how large an error occur due to the euler approximation and if this error is of any significants to the predictions made by the model. If sow would some other method of approximation give rise to any significant improvements.

Another interesting investigation could consist of making predictions and evaluating actual designs such ass malls, theaters etc. Then investigate the predictions made by the model, compared to the fire marshals actual recommendations concerning the number of people allowed in the room, the number of fire exists, how much time should it take to clear the room, etc. Does the predictions made by the models correspond to real life recommendations?