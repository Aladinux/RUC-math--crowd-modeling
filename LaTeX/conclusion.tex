% vim:ft=tex
\section{Summary and conclusion}
\label{sec:conclusion}

We have created a computer simulation of a social force model in order to 
study social force models in general and find out how well social force models 
simulate the behaviour of crowds, and what strengths and weaknesses of these 
models are.

In order to get an overview of the different social force models, we have 
first examined the field of social force modelling and presented the 
variations that exist, as well as some of the results that are presented in 
the literature. We have picked an exemplary model that is described in detail 
in the article that presents it, and analysed it in detail. In this process we 
have had to fill in details from articles other than the one that presents our 
chosen model.

Based on the analysis of the model, we have implemented our simulation of the 
model. While doing this, we have had to fill in some details in order to go 
from the abstract model formulation to a concrete numerical simulation. These 
details include how to approximate the movement of pedestrians, how to set 
initial conditions and values, and how to implement the interaction between 
pedestrians and walls in practice.

From our results, it is clear that we have successfully implemented a running 
model simulation, that (with the right parameters) exhibits reasonable 
pedestrian behaviour upon visual inspection. We have not done an exhaustive 
review of the different parameters' effect on the simulations, however, since 
we have deemed doing so manually impractical, and we have not been able to 
come up with a way to automate it. As such, we are not able to say anything 
conclusive about the parameters' effect on model results.

We have attempted to replicate the results we have seen in the literature.  
While some results have been successfully replicated, other effects do not 
manifest themselves in our results. We have discussed several reasons for this 
discrepancy, including features that are lagging from the model, parameter 
values, effects of using random numbers to generate the initial conditions and 
possible errors in our implementation of the model.

Based on the results of our own simulations and our review of the social force 
modelling field, we have assessed social force models and their strengths and 
weaknesses. We conclude that social force models are not built on any theories 
for the behaviour of crowds, but are created to replicate a set of 
observations. As such, any confidence in their predictions must come from a 
record of producing results fitting observations; and since the field is 
relatively new, they have not yet reached this state. Social force models do, 
however, provide a practical way to simulate something that would otherwise be 
impossible to simulate. As such, they are the best available way to provide 
e.g. guidance when designing facilities that must accommodate many 
pedestrians, and given time the accuracy of their predictions will probably 
increase.


\subsection{Further work}
We see three main directions our work could be continued in: adding new 
features to the model, examining parameter interdependence and the effects of 
different parameter values on the model behaviour and testing our model 
implementation on real world observations.

In section~\ref{sec:new-forces}, we discuss features that we believe would 
improve the results obtained from the simulations if added to the model. A 
possible avenue for further work would be adding some of these features and 
examining what impact, if any, they have on the model behaviour. Among the 
evaluated features could be substituting another way to approximate pedestrian 
movement than using Euler's method as we do.

Another possible expansion of our work is coming up with a way to 
automatically test whether a simulation run is viable (i.e. doesn't break down 
due to pedestrians walking through walls etc.), and using this to test 
different parameter settings to discover which parameters are interdependent, 
and how different parameter setting affect the viability of the simulations.

Finally, we think it would be interesting to compare the results of our 
simulations with real world observations. While some articles compare 
the results of simulations with real world observations, these comparisons are 
specific to the simulation implementations used in the article (the details of 
which are not always available). We think it would be interesting to test our 
implementation of the simulation against real world observations.
