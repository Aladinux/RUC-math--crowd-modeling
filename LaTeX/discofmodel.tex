\section{Discussion of model}
In this section we will make an analysis of the different parts and 
parameters of the model and see what their role in the model is as well 
as inspect their domains and limitations. 

\subsection{The time steps}
In this section discuss how to set the the time steps of the calculation. 
If it is too big we risk that agents might end on top of each other and 
our model can not handle this kind of situation however if we set the time 
steps to be extremely low the computation time will be extremely large. 
How should we decide to set the time steps such that we won't risk the model 
breaking down and we won't get too long computation time?

\subsection{The fluctuation term}
In addition to all the forces action on the agent the change in velocity of agent 
$\alpha$ is controlled by fluctuation term. The role and nature of this term will 
be discussed in this subsection.

\subsection{The impatience/nervousness factor}
The impatience or nervousness factor is active when one calculates the force action 
on agent $\alpha$ from the velocity of the agent. This means that the agents earlier 
velocity alone will take a part in deciding the future velocity of the agent. This 
force is controlled by the impatience or nervousness factor. The impatience or 
nervousness factor is as stated earlier given by:

\begin{equation}
	\eta_{\alpha} \left( t \right) =
    1 - \frac{\overline{V}_{\alpha} \left( t \right)}
             {V_{\alpha}^{0} \left( t \right)}
\end{equation}

where $\overline{V}$ is the average speed in the desired direction and as 
earlier $V_{\alpha}^{0} \left( 0 \right)$ is the speed at the beginning of the 
first calculation step of the simulation.

The impatience or nervousness factor i directly seen in the expression for 
$V_{\alpha}^{0}$.

\begin{equation}
    V_{\alpha}^{0} = \left[ 1 - \eta_{\alpha} \left( t \right) \right] 
    V_{\alpha}^{0} \left( 0 \right) +
    \eta_{\alpha} \left( t \right)V_{\alpha}^{\text{max}}
\end{equation}

In the case where $0 \leq \eta_{\alpha} \leq 1$ the expression for 
$V_{\alpha}^{0} \left( t \right)$  makes sense. Here we can see why this term 
is called the impatience of the agent. If the fraction  between the average 
speed in the desired direction and the initial speed is low then $\eta_{\alpha} \approx 1$. 
When the impatience term is close to one $V_{\alpha}^{0} \left( t \right)$ 
is dominated by $V_{\alpha}^{\text{max}}$. That is, if the agent have not 
moved very far in the desired direction compared to the initial speed the 
impatience of the agent will cause the agent's future velocity to be dominated by 
the desired velocity of the agent.

If the agent has been moving in the desired direction with his initial 
speed the entire time then $\eta_{\alpha} = 0$  and 
$V_{\alpha}^{0} \left( t \right)$ will continue to be $V_{\alpha}^{0} \left( 0 \right)$.

In the case where $\eta_{\alpha} \leq 0$ that is the agent has moved further 
in the desired direction then he would have had he been walking with his 
initial speed. The expression for $V_{\alpha}^{0} \left( t \right)$
stats yield strange results. That $\eta_{\alpha} \leq 0$ would imply that:

\begin{equation}
    V_{\alpha}^{0} = \left[ 1 + \eta_{\alpha} \left( t \right) \right] 
    V_{\alpha}^{0} \left( 0 \right) -
    \eta_{\alpha} \left( t \right)V_{\alpha}^{\text{max}}
\end{equation}

And this will yield a negative value for $V_{\alpha}^{0}$ if $\left[ 1 + \eta_{\alpha} \left( t \right) \right] 
V_{\alpha}^{0} \left( 0 \right) < \eta_{\alpha} \left( t \right)V_{\alpha}^{\text{max}}$. 
This is a problem because it is not that far fetched that an agent will be 
forced to exceed his desired velocity.

In the case where $1 \leq \eta_{\alpha}$ it would mean that the agent has moved 
further in the opposite direction than the desired one and this can only happen very 
weird situations.

\subsection{The force between agents}
When calculating the force between agents you need to have a normal vector 
pointing from one agent to another. However if agents somehow end on top of 
each other the model breaks down because then the normal vector will be pointing 
in a strange direction.