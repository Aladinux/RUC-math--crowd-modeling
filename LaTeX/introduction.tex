% vim:ft=tex
\section{Introduction}
When a lot of people are gathered in a confined space, it is not immediately 
obvious how they behave when moving around. It is also not obvious how to 
design buildings and other areas where many people gather to allow for optimal 
passage of crowds without jamming and related problems (such as inconvenience 
for pedestrians and even injuries). It is cumbersome, or even dangerous, to 
gather many people together to do experiments, especially if one wishes to 
simulate a panic situation (which cannot necessarily be induced artificially).  
It would thus be useful to be able to model a crowd of pedestrians and their 
movement.

A crowd of many people is a very complex system. One of the approaches to 
handling this complexity is by using a computer simulation of a model derived 
from the behaviour of individual pedestrians, and through the outcome of this 
simulation be able to say something meaningful about the whole system. This 
approach to modelling complex systems is called agent-based modelling and is 
used in many different and diverse subject areas, including molecular biology, 
economics and physics.

Within the field of crowd simulations, there  are various ways of formulating 
such an agent-based model; we have chosen to focus on an approach that models 
pedestrian behaviour using a concept called ``social forces'' 
\cite{social-force}. This type of model describes individual behaviour 
borrowing concepts and notation from physics, by defining  virtual ``forces'' 
representing desired movement as well as the tendency to avoid obstacles, 
other people etc. Through computer simulations this can be used to say 
something about the whole system of moving pedestrians.

Since the original formulation of the social force concept, it has been 
revised several times (e.g.  \cite{helbing00}). For this project we base our 
analysis on the  version presented in \cite{self-org}, as this article goes 
into a reasonable level of detail. We will, however, also include other 
variants where necessary.  Our chosen model, and others like it, describe a 
set of properties of crowd behaviour that has been seen in observations of 
actual crowds as well as replicated in computer simulations.  We think it will 
be interesting to go into detail with this type of model and look into how 
they work and what they can tell us about crowds. This brings us to our 
problem formulation:

\subsection{Problem formulation}
\begin{quote}
    How do social force models work and what do they say about crowds?

    What are the assumptions underlying social force models, and what are 
    their strengths and limitations?
\end{quote}

Our problem is exemplary to mathematical modelling for several reasons. First 
of all, it is an example of applying mathematical theory to a field outside of 
mathematics itself. The field we are working with (crowd simulation) is 
relatively new, but has some established research\footnote{See the overview of 
the field in section~\ref{sec:social-forces}.}. The model itself contains a 
non-trivial application of mathematics and is an example of a type of model 
(agent-based models) used in many cases to study a complex system of 
interacting pedestrians by describing the individual behaviour of the pedestrians and 
using simulations to derive meaningful properties for the whole system. These 
factors combined makes our problem exemplary.
% TODO: References for agent-based models.

\subsection{Target audience}
The target audience of this report is comprised of students of mathematics on 
a level of education similar to our own. This means that we consider 
mathematical concepts and theory that is covered in the bachelor courses at 
the mathematics department at RUC as known subject matter, and will therefore 
not explain these concepts in detail. Concepts that are beyond this are 
explained as necessary. In the description of the simulation program code, 
some familiarity with basic constructs of computer programming is assumed, and 
are thus not explained.

\subsection{Approach to answering the problem formulation}
To answer the problem formulation, it is first necessary to establish the 
concept of social force models in more detail, as well as give an overview 
over the work in the field. This is presented in 
section~\ref{sec:social-forces}, where we also present the model we have chosen 
to focus on, including the results presented in it.

In order to better understand the model, we are going to make our own computer 
simulation of it. To do so, we will first describe the model thoroughly. In 
section~\ref{sec:the-model}, we will go over the different parts and 
parameters of the model, and point out parts that are ambiguous or missing 
from the article presenting the model.  Based on this description, we will 
discuss what is needed to turn the model into a numerical simulation, and how 
we have done so, in section~\ref{sec:model-to-simulation}.

Based on our analysis of the model, we will construct a computer simulation of 
it, modelling cases analogous to those presented in 
section~\ref{sec:social-forces}. Our implementation of the 
simulation is discussed in section~\ref{sec:simulation}, including set up of 
parameters and initial conditions, the structure of our simulation and how we 
obtain results from it.

The actual results of the simulation is addressed in 
section~\ref{sec:results}. Here we present the outcome of our simulation, and 
whether or not we have been able to replicate the results we were aiming for.  
Based on this presentation, we discuss in section~\ref{sec:discrepancies} 
possible reasons for any discrepancies between our own results and the results 
we are trying to replicate, as well as which alterations of the model would 
be needed to fix them.

Based on our results and the proposed changes to the model, we will assess 
social force models in general in section~\ref{sec:assessment}, giving an 
overview of the strengths and weaknesses of this modelling approach. Finally, 
we will give a summary of our findings and an overview of the whole project in 
section~\ref{sec:conclusion}.
