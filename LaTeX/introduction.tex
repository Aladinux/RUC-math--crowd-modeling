% vim:ft=tex
\section{Introduction}
When a lot of people are gathered in a confined space, it is not immediately 
obvious how they behave when moving around. It is also not obvious how to 
design buildings and other areas where many people gather to allow for optimal 
passage of crowds without jamming and related problems (such as inconvenience 
for pedestrians and even injuries). It is cumbersome, or even dangerous, to 
gather many people together to do experiments, especially if one wishes to 
simulate a panic situation (which cannot necessarily be induced artificially).  
It would thus be useful to be able to model a crowd of pedestrians and their 
movement.

A crowd of many people is a very complex system. One of the approaches to 
handling this complexity is by using a computer simulation of a model derived 
from the behaviour of individual actors, and through the outcome of this 
simulation be able to say something meaningful about the whole system. There 
are various ways of formulating such a model; among them is an approach that 
models pedestrian behaviour using a concept called ``social forces'' 
\cite{social-force}. This type of model describes individual behaviour using a 
set of forces representing desired movement as well as the tendency to avoid 
obstacles, other people etc. Through computer simulations this can be used to 
say something about the whole system of moving pedestrians.

The social force class of models represent an agent-based approach to 
modelling crowds that we think is interesting. Since the original formulation 
of the social force concept, it has been revised several times (e.g.  
\cite{helbing00}). For this project we have selected the version presented in 
\cite{self-org}.  We think it will be interesting to look into what this model 
makes it possible to say about the dynamics of a (virtual) crowd. This brings 
us to our problem formulation:

\subsection{Problem formulation}
\begin{quote}
    What does the model presented in \cite{self-org} tell us about the 
    behaviour of crowds?

    Which concepts are meaningful when describing the behaviour of crowds?

    What are the properties of the chosen model with respect to the chosen 
    crowd behaviour concepts, and how do the model parameters affect these 
    properties?
\end{quote}

Our problem is exemplary to mathematical modelling for several reasons. First 
of all, it is an example of applying mathematical theory to a field outside of 
mathematics itself. The field we are working with (crowd simulation) is 
relatively new, but has some established research\footnote{See the overview of 
the field in section~\ref{sec:the-model}.}. The model itself contains 
non-trivial mathematics and is an example of a type of model (agent-based 
models) used in many cases to study a complex system of interacting agents by 
describing the individual behaviour of the agents and using simulations to 
derive meaningful properties for the whole system\footnote{See the overview of 
agent-based models in section~\ref{sec:the-model}.}. These factors combined 
makes our problem exemplary.

\subsection{Approach to answering the problem formulation}
To answer the problem formulation, it is first necessary to establish some 
concepts for discussing the behaviour of crowds in general, in order to have a 
basis on which we can analyse our chosen model. This, along with the cases 
that we wish to simulate, is described in section~\ref{sec:crowds}. This 
section also contains an overview of the work that has been done in the field 
of crowd modelling and which approaches there has been to tackling this area, 
as well as background information on the type of model (agent-based) we are 
dealing with.

In order to subject the model to further analysis, we will first describe it 
thoroughly. In section~\ref{sec:the-model}, we will go over the different 
parts and parameters of the model, and point out interesting areas to be 
discussed further.  Based on this description, we will go into a detailed 
discussion of interesting areas in section~\ref{sec:discussion}.
This includes discussing how the chosen areas affect the behavior of the model and 
how we have interpreted any ambiguities we may encounter in the model.

Based on our analysis of the model, we will construct a computer simulation of 
the model, modelling the cases we have selected. Our implementation of this 
simulation is discussed in section~\ref{sec:simulation}, including set up of 
parameters and initial conditions, the structure of our simulation and how we 
obtain results from it.

The actual results of the simulation is addressed in 
section~\ref{sec:results}. Here we discuss how the simulation results relate 
to the concepts for crowd behaviour we have chosen to focus on, and what this 
tells us about crowd behaviour in the real world, as well as which limitations 
these results have.  We will give an assessment of how well the model fits 
empirical data, by examining results obtained by others in the literature, 
rather than by doing our own empirical study.

Finally, we conclude on our findings in section~\ref{sec:conclusion} and touch 
briefly upon the perspectives for this area of study as well as further things 
that might be investigated from our results in section~\ref{sec:perspectives}.
