% vim:ft=tex
\section{Introduction}
When a lot of people are gathered in a confined space, it is not immediately 
obvious how they behave when moving around. It is also not obvious how to 
design buildings and other areas where many people gather to allow for optimal 
passage of crowds without jamming and related problems (such as inconvenience 
for pedestrians and even injuries). It is cumbersome, or even dangerous, to 
gather many people together to do experiments, especially if one wishes to 
simulate a panic situation (which cannot necessarily be induced artificially).  
It would thus be useful to be able to model a crowd of pedestrians and their 
movement.

A crowd of many people is a very complex system. One of the approaches to 
handling this complexity is by using a computer simulation of a model derived 
from the behaviour of individual actors, and through the outcome of this 
simulation be able to say something meaningful about the whole system. There 
are various ways of formulating such a model; among them is an approach that 
models pedestrian behaviour using a concept called ``social forces'' 
\cite{social-force}. This type of model describes individual behaviour using a 
set of forces representing desired movement as well as the tendency to avoid 
obstacles, other people etc. Through computer simulations this can be used to 
say something about the whole system of moving pedestrians.

The social force class of models represent an agent-based approach to 
modelling crowds that we think is interesting. Since the original formulation 
of the social force concept, it has been revised several times (e.g.  
\cite{helbing00}). For this project we have selected the version presented in 
\cite{self-org}, because this version has been expanded to include features 
not available in earlier models\footnote{This should probably be expanded.}.  
We think it will be interesting to look into what this model makes it possible 
to say about the dynamics of a (virtual) crowd. This brings us to our problem 
formulation:

\subsection{Problem formulation}
\begin{quote}
    What does the model presented in \cite{self-org} tell us about the 
    behaviour of crowds?

    Which concepts are meaningful when describing the behaviour of crowds?

    What are the properties of the chosen model with respect to the chosen 
    crowd behaviour concepts, and how do the model parameters affect these 
    properties?
\end{quote}

\subsection{Approach to answering the problem formulation}
We will seek to answer our problem formulation by subjecting our chosen model 
to thorough analysis, determining which properties are used in the description 
of the model to describe its behaviour. This analysis will include a review of 
the model's parameters, where they come from, and what role they play in the 
model as well as an assessment of what kind of model behaviour we expect from 
the model.

Based on this analysis, we will create a computer simulation of a virtual 
scenario using the model. From this simulation we will illustrate the 
properties of the model, and examine how the model parameters affect these 
properties by varying the parameters in the simulation.

We will give an assessment of how well the model fits empirical data, by 
examining results obtained by others in the literature, rather than by doing 
our own empirical study. Based on this, and our simulation, we will assess how 
well this model is suited to modelling a crowd, and what its limitations are.

\subsection{Exemplarity}
Our project is exemplary to mathematical modelling for several reasons. First 
of all, it is an example of applying mathematical theory to a field outside of 
mathematics itself. The field we are working with (crowd simulation) is 
relatively new, but has some established research. The model itself contains 
non-trivial mathematics and is an example of a type of model (agent-based 
models) used in many cases to study a complex system of interacting agents by 
describing the individual behaviour of the agents and using simulations to 
derive meaningful properties for the whole system. These factors combined 
makes our project exemplary.
