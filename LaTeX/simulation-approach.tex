% vim:ft=tex
\section{Simulation approach}
In this section, we describe how our simulation is implemented, and how the 
implementation works.

Our simulation is implemented in the Python programming language, using the 
\emph{numpy} numerical calculation library. We assume a virtual coordinate 
system using meters as a base unit, and with the origin in the lower left 
corner of the area we simulate. Each pedestrian (or actor) is described by a 
centre point and a radius, and each wall is described by a line segment 
connecting two points.

\subsection{Initial conditions}
The initial conditions are the following:

\begin{itemize}
    \item A position, $P_n$, a velocity vector, $V_n$, and a desired direction 
        vector, $V^D_{n}$ for each actor.
    \item A position, $W_n$ for each wall.
\end{itemize}

The walls are set up to form the room we want to simulate, and the desired 
destinations are set outside the exit(s) for all actors. The actors are 
distributed throughout the room and given initial velocities and desired 
walking speed towards the exit by means of some kind of 
distribution\footnote{Further explanation to be added.}.

Also, there is a time step parameter, set initially to $0.1$ seconds. And some 
constants set arbitrarily.
%TODO: Add real description of initial conditions.

\subsection{The simulation steps}
The simulation is divided into two parts: Finding the accelerations (or 
resulting force) for all actors, and updating position and velocity for the 
actors.  Since the acceleration for each actor is dependent on both position 
and velocity of the other actors, splitting the calculations this way enables 
us to do the calculations of each actor in any order, and even parallel. The 
drawback is that the actors are only affected by the movement and positions of 
other actors as they were at the end of the last simulation step. This means 
that the time step has to be small enough that this doesn't matter in 
practice.

\subsubsection{Calculating the acceleration vectors}
The acceleration vectors for each actor is calculated as follows:

\begin{enumerate}
    \item Determine the acceleration vector provided by the actors desired 
        direction of movement.
    \item For each other actor, calculate the acceleration vector provided by 
        avoiding the actor.
    \item Calculate avoidance accelerations form the walls.
    \item Sum all these vectors.
\end{enumerate}

% start parameters: p_n, v_n for each person.
% p for each wall
%
% steps:
%
% for each person:
%   calculate acceleration from the force parts of the model
%
% for each person:
%   update position by displacement vector
%       delta-p = (v_x*delta-t+1/2*a_x*delta-t^2, 
%       v_y*delta-t+1/2*a_y*delta-t^2)
%   update velocity for next step
%
%
% initial parameters:
%   Set using some sort of distribution around a mean
%
%
% constants:
%   Where do they come from?
