% vim:ft=tex
\section{Simulation approach}
\label{sec:simulation}
In this section, we describe how our simulation is implemented, and how the 
implementation works. This is done in part to document our implementation, and 
in part to give the reader an overview of how our results are obtained.
We will describe briefly how the program is structured on a macro level, and 
go into more detail on the parts specific to the calculations of the model 
parameters. The full source code is available online\footnote{See 
\url{http://akira.ruc.dk/~tohojo/crowd-modelling}.}.

Our simulation is implemented in the Python programming language, with the 
calculation intensive parts implemented in C for performance reasons. We 
assume a virtual coordinate system using meters as a base unit, and with the 
origin in the centre of the area we simulate. Each pedestrian is 
described by a centre point and a radius, and each wall is described by a line 
segment connecting two points.

All parameters are stored as double precision floating point values where 
nothing else is indicated. We use custom data structures to keep track of the 
pedestrians and walls while running the simulation. Python is used to set up the 
initial conditions, run the program's main control loop, and draw the 
simulation results through the \emph{PyGame} library \cite{pygame}. This 
allows to do real-time animation as well as saving each simulation step 
to be assembled into a film afterwards. All calculations and data processing 
is done in a Python extension written in C, to increase performance. Gathering 
of data to draw the graphs and the drawing itself is done in the Python code. 

\subsection{Structure of the program}
The program is structured into four main parts: The simulation calculations, 
drawing of the simulations, plotting of parameters and the control part 
setting up parameters and calling the other parts as necessary. The drawing 
part mainly consists of a frontend to the drawing library, and so is not 
interesting to discuss here. The other parts will be described in the 
following, structured so as to present the sequence that is followed when a 
simulation is run. This description consists of three parts: setting up the 
simulation, calculations, and gathering of results.

\subsection{Setting up the simulation}
The program supports defining multiple \emph{scenarios} to simulate. Each 
scenario defines its own set of parameters, and a simulation run features one 
scenario. The parameters defined for each scenario include:

\begin{itemize*}
    \item The model constants, $A$, $B$, $U$, $\lambda$ and pedestrian relaxation 
        time. These are the same for all pedestrians in a simulation.
    \item Mean initial desired velocity and radius of pedestrians and their standard 
        deviation, and the factor used to calculate the maximum velocity.
    \item The initial number of pedestrians, the area(s) they start in and the 
        target(s) they move towards.
    \item The geometry of the scenario (i.e. placement of walls).
    \item Definition of the areas where measurement of data for plotting 
        graphs is done and which parameters should be plotted (see 
        section~\ref{sec:measurement}).
    \item Various parameters related to drawing of the simulation, maximum run 
        time and optional continuous inflow of pedestrians.
\end{itemize*}

Parameters that are not set for each scenario, but are defined once for all 
simulations are:

\begin{itemize*}
    \item Time step size.
    \item Plotting data sampling frequency.
\end{itemize*}

How the values of the parameters are determined is described in 
section~\ref{sec:init-cond}. From these parameters, the simulation is set up 
by initialising the calculation module and creating the pedestrians.

The pedestrians are created with an initial velocity of zero, and distributed 
randomly within the area(s) designated by the parameters for the scenario, as 
described in section~\ref{sec:init-pedestrians}. 

\subsection{Calculation of the model}
\label{sec:model-calculation}
For each step of the simulation, the calculations are run in two parts: 
Finding the accelerations (or resulting force) for all pedestrians, and 
updating position and velocity for the pedestrians.  Since the acceleration 
for each pedestrian is dependent on both position and velocity of the other 
pedestrians, splitting the calculations this way enables us to do the 
calculations of pedestrians in any order, and even parallel. The drawback 
is that the pedestrians are only affected by the movement and positions of 
other pedestrians as they were at the end of the last simulation step. This 
means that the time step has to be small enough that this doesn't matter in 
practice.

The acceleration vectors for each pedestrian, $\alpha$, is calculated in three 
steps, corresponding to the parts $\overrightarrow{f_\alpha^d}$, 
$\overrightarrow{f_{\alpha \beta}}$ and $\overrightarrow{f_{\alpha \gamma}}$ from 
section~\ref{sec:the-model}. In each simulation step the three forces are 
calculated in order, first calculating the desired force, then the repulsive 
force from each of the other pedestrians in the simulation, and finally the 
repulsive force from each wall. The resulting vectors are added to yield the 
total acceleration for each pedestrian.

After this is calculated for all pedestrians, their velocities and positions are 
updated in separate steps. Each of these four steps are described in detail 
in the following. The code calling the other parts of the calculation can be 
seen in code segment~\ref{lst:calling}. This function is called once for each 
pedestrian (index \texttt{i}) for each simulation step.

\begin{lstlisting}[caption={Main function calling the other parts of the 
    calculation code.},label=lst:calling]
void calculate_forces(Py_ssize_t i)
{
    int j;
    add_desired_acceleration(&pedestrians[i]);

    for(j = 0; j < a_count; j++) {
        if(i == j) continue;
        add_repulsion(&pedestrians[i], &pedestrians[j]);
    }

    add_wall_repulsion(&pedestrians[i]);
}
\end{lstlisting}

\subsubsection{Calculating the desired force}
The function for calculating the desired force can be seen in code 
segment~\ref{lst:desired-force}. Of note is the implementation on lines 7-19 
of the conditional definition of $V_\alpha^d(t)$ given in 
equation~\eqref{eqn:cond-define} and the creation of a unit vector in lines 
20-21 to add the direction to the desired velocity.

To increase performance, it is preferred to update values in place, instead of 
copying them around in different parts of the computer's memory. This gives 
rise to the constructs around the \texttt{vector\_i*}-functions in the code; 
here the \emph{i} stands for ``in place''.

\begin{lstlisting}[caption={Calculating the desired 
    force.},label=lst:desired-force]
static void add_desired_acceleration(Pedestrian * a)
{
    double average_velocity = 0.0, impatience = 0.0, 
           desired_velocity = 0.0;
    Vector desired_direction = {0.0, 0.0};

    if(a->time) {
        double proj = vector_projection_length(
                a->initial_position, a->target, a->position);
        average_velocity = proj / a->time;

        impatience = 1.0 - average_velocity / a->initial_desired_velocity;

        desired_velocity = (1.0-impatience) * a->initial_desired_velocity + \
                           impatience * a->max_velocity;

    } else {
        desired_velocity = a->initial_desired_velocity;
    }
    desired_direction = vector_sub(a->target, a->position);
    vector_unitise(&desired_direction);

    a->acceleration = vector_mul(desired_direction, desired_velocity);
    vector_isub(&a->acceleration, &a->velocity);
    vector_imul(&a->acceleration, 1.0/a->relax_time);
}
\end{lstlisting}

\subsubsection{Repulsion from other pedestrians}
The two functions used for calculating the repulsion from other pedestrians is seen 
in code segment~\ref{lst:pedestrian-repulsion}. The \texttt{add\_repulsion} function 
is called for each pair of pedestrians. The force without any angle dependence is 
calculated in the \texttt{calculate\_repulsion} function, and is modified by 
the angular dependence if the pedestrian's velocity is not zero, and $\lambda$ is 
not one. If the velocity is zero, calculating the dot product would result in 
a division by zero, and if $\lambda$ is one, modifying the force by the 
angular dependence would have no effect. The whole calculation is skipped if 
one of the constants $A$ or $B$ are zero, as this would make the calculations 
have no effect, or result in a division by zero, respectively.

\begin{lstlisting}[caption={Calculating the repulsion from other 
    pedestrians.},label=lst:pedestrian-repulsion]
Vector calculate_repulsion(Pedestrian * a, Pedestrian * b, double A, double B)
{
    double radius_sum = a->radius + b->radius;
    Vector from_b     = vector_sub(a->position, b->position);
    double distance   = vector_length(from_b);

    vector_unitise(&from_b);
    vector_imul(&from_b, A * exp((radius_sum-distance)/B));

    return from_b;
}

void add_repulsion(Pedestrian * a, Pedestrian * b)
{
    if(!A || !B) return;
    Vector repulsion = calculate_repulsion(a, b, A, B);
	if(a->velocity.x && a->velocity.y && lambda < 1.0) {
		Vector from_a = vector_sub(b->position, a->position);

		double cosine = vector_dot(a->velocity, from_a)/(
				vector_length(a->velocity) * vector_length(from_a));

		vector_imul(&repulsion, (lambda + (1-lambda)*((1+cosine)/2)));
	}
    vector_iadd(&a->acceleration, &repulsion);
}
\end{lstlisting}

\subsubsection{Repulsion from walls}
The code for adding the repulsion from the walls is shown in code 
segment~\ref{lst:wall-repulsion}. The \texttt{add\_wall\_repulsion} function 
is called for each pedestrian and consists of two parts: Finding the points on the 
walls where repulsion should be calculated from, and calculating the repulsion 
from these points. The force from the repulsion points is calculated in the 
function \texttt{calculate\_wall\_repulsion} that is called for each repulsion 
point. This calculation corresponds to equation~\eqref{eqn:wall-repulsion}.

\begin{lstlisting}[caption={Calculating the repulsion from the 
    walls.},label=lst:wall-repulsion]
void add_wall_repulsion(Pedestrian * a)
{
    int i;
    Vector * repulsion_points  = PyMem_Malloc(w_count * sizeof(Vector));
    int rep_p_c = 0;
    Vector repulsion;

    rep_p_c = find_repulsion_points(a, repulsion_points);

    for(i = 0; i < rep_p_c; i++) {
        repulsion = calculate_wall_repulsion(a, repulsion_points[i]);
        vector_iadd(&a->acceleration, &repulsion);
    }


    PyMem_Free(repulsion_points);
}

Vector calculate_wall_repulsion(Pedestrian * a, Vector repulsion_point)
{
    Vector repulsion_vector = vector_sub(a->position, repulsion_point);
    double repulsion_length = vector_length(repulsion_vector);
    vector_unitise_c(&repulsion_vector, repulsion_length);

    double repulsion_force = (1/a->radius) * U * exp(-repulsion_length/a->radius);

    vector_imul(&repulsion_vector, repulsion_force);

    return repulsion_vector;
}
\end{lstlisting}

The function for finding the repulsion points on the walls is seen in code 
segment~\ref{lst:repulsion-points}. This is an implementation of the algorithm 
described in section~\ref{sec:repulsion-points}. Lines 9-27 corresponds to 
step one in the algorithm description that identifies definite repulsion 
points (those that are not wall endpoints) and possible repulsion points 
(those that are endpoints).

In lines 29-35 an endpoint is discarded if it is already used because it is 
shared with a wall that has already defined a repulsion point. In lines 36-57 
endpoints that are not discarded in this way are added to the list of 
repulsion points if they are either not free-floating (i.e. shared with 
another wall), or closer to the pedestrian centre than the pedestrian's 
radius. This ensures that doorways do not repulse pedestrians that are passing 
through it.

\begin{lstlisting}[caption={Finding the wall repulsion 
    points.},label=lst:repulsion-points]
int find_repulsion_points(Pedestrian * a, Vector repulsion_points[])
{
    int i,j;
    double projection_length;
    Vector * used_endpoints     = PyMem_Malloc(2*w_count * sizeof(Vector));
    Vector * possible_endpoints = PyMem_Malloc(w_count * sizeof(Vector));
    int rep_p_c = 0, use_e_c = 0, pos_e_c = 0;

    for(i = 0; i < w_count; i++) {
        Wall w = walls[i];
        projection_length = vector_projection_length(w.start, w.end, a->position);
        if(projection_length < 0)  {
            possible_endpoints[pos_e_c++] = w.start;
        } else if(projection_length > w.length) {
            possible_endpoints[pos_e_c++] = w.end;
        } else {
            // We have the length, L, of how far along AB the projection point is.
            // To turn this into a point, we multiply AB with L/|AB| and add
            // this vector to the starting point A.
			// P = A + AB*L/|AB|
            repulsion_points[rep_p_c++] = vector_add(w.start, 
                    vector_mul(vector_sub(w.end, w.start), 
                        projection_length/w.length));
            used_endpoints[use_e_c++] = w.start;
            used_endpoints[use_e_c++] = w.end;
        }
    }

    for(i = 0; i < pos_e_c; i++) {
        int use_e = 1;
        for(j = 0; j < use_e_c; j++) {
            if(vector_equals(possible_endpoints[i], used_endpoints[j])) {
                use_e = 0;
            }
        }
        if(use_e) {
			// Keep track of whether the endpoint is free-floating, i.e. if
			// it is shared with another wall.
			int free_e = 1;
			for(j = 0; j < pos_e_c; j++) {
				if(i != j && 
						vector_equals(possible_endpoints[i],
							possible_endpoints[j])) {
					free_e = 0;
				}
			}
			// Endpoints that are free-floating (i.e. sides of doorways) are
			// only considered for repulsion if they are closer to the pedestrian
			// than the pedestrian's radius. This allows pedestrians to pass more
			// freely through doorways.
			if(!free_e || 
					vector_length(vector_sub(a->position,
							possible_endpoints[i])) < a->radius) {
				repulsion_points[rep_p_c++] = possible_endpoints[i];
				used_endpoints[use_e_c++] = possible_endpoints[i];
			}
        }
    }

    PyMem_Free(used_endpoints);
    PyMem_Free(possible_endpoints);

    return rep_p_c;
}
\end{lstlisting}

\subsubsection{Updating position and velocity}
After every pedestrian has been updated with a resulting acceleration vector from 
the current simulation step, all pedestrians update their position and velocity.  
The position is updated by calculating a displacement vector as explained in 
section~\ref{sec:continuous-movement}.

After updating the position, the pedestrian's velocity is updated by adding 
the acceleration vector, to be used for the next simulation step. The code 
that does this is seen in code segment~\ref{lst:update-position}. Line 14 is 
related to the measuring of data for plotting, and will be explained in 
section~\ref{sec:measurement}.

\begin{lstlisting}[caption={Updating the pedestrian 
    position.},label=lst:update-position]
void update_position(Pedestrian * a)
{
	Vector delta_p = vector_add(
			vector_mul(a->velocity, timestep), 
			vector_mul(a->acceleration, 0.5 * pow(timestep, 2)));

    vector_iadd(&a->position, &delta_p);
	a->velocity = vector_add(a->velocity,
			vector_mul(a->acceleration, timestep));
    a->time += timestep;

    check_flowlines(a);
}
\end{lstlisting}

\subsection{Obtaining results}
\label{sec:measurement}
There are two ways we obtain results from the simulation: through the drawings 
of the simulations themselves, and through graphs of various measurements made 
during the simulation. In this section we describe the mechanisms for 
obtaining these results. Which results are used for which scenarios are 
described, along with the scenarios themselves, in section~\ref{sec:results}.

\subsubsection{Drawings of the simulation}
When the simulation is run, drawings of the pedestrian's positions, walls etc. are 
created for each simulation step. These can be shown in real time, or saved 
and either assembled into a film or used as individual illustrations. We use 
these images to make qualitative assessments of pedestrian behaviour, e.g. 
determining whether a lane formation occurs. We include the relevant images as 
illustrations in the results section.

\subsubsection{Graphs of measurements}
As part of the simulation program, we have added measurements of various 
properties of the simulation, which are then assembled into graphs after the 
simulation has run. This section describes the measuring features we have 
added to the program, and \emph{how} they work. \emph{Why} we measure each 
property is explained along with the results. Since we have revised which 
graphs are needed throughout our work with the simulations, some additional 
measuring and graphing features exist in the source code, but are not 
described here.

The graphing functionality works in two different modes: One is graphing 
properties as a function of the simulation time, and others run multiple 
simulations varying one or more parameters and graphing properties as a 
function of this varied parameter. Any numeric parameter can be varied in this 
way. It is noted for each feature in which mode it is used and some properties 
are used for both modes. The properties we measure are:

% TODO: Pros and cons of these measurements.

\begin{itemize}

    \item \textbf{Flow rate:} Flow rate is measured along a line segment 
        defined as a parameter to the scenario; the line segment is assumed to 
        be perpendicular to the primary movement direction of the pedestrians.  
        The simulation keeps track of at what time each pedestrian first comes 
        into contact with this line segment, i.e. is first less than their 
        radius away from it . This is used as an approximation of when the 
        pedestrians cross the line. The average flow rate is then computed as 
        the number of pedestrians who have crossed the line divided by the 
        time.  This is graphed as a function of time, using a five second 
        moving average, and as a function of parameters, using the average of 
        the flow rate measurements of an entire simulation.

    \item \textbf{Leaving time:} The leaving time is a measure of how long it 
        takes for all pedestrians to leave the simulation completely. Pedestrians are 
        removed from the simulation when they reach their target, or when they 
        get outside the calculation area which is the basis of this 
        measurement. This property is only measured for a whole simulation, 
        and as such is only graphed as a function of the varying parameters. 
        Of course simulations that do not have a finite running time (i.e. 
        where pedestrians are added continuously, or where they never reach their 
        target) do not measure this property.
\end{itemize}

\subsection{Summary}
In this section we have described our implementation of the simulation. The 
program is split up into three parts: setting up the simulation with 
parameters and initial conditions, the calculations of the model itself, and 
measurements of the results. The first and last parts are implemented in the 
Python programming language, while the calculations themselves are implemented 
in C for performance reasons. The measurements of simulation results include 
drawing of pictures of the simulation, and graphs of the flow rate and the 
time it takes all pedestrians to leave a room.
