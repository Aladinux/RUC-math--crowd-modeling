% vim:ft=tex
\section{Simulation approach}
In this section, we describe how our simulation is implemented, and how the 
implementation works.

Our simulation is implemented in the Python programming language, using the 
\emph{numpy} numerical calculation library. We assume a virtual coordinate 
system using meters as a base unit, and with the origin in the lower left 
corner of the area we simulate. Each pedestrian (or actor) is described by a 
centre point and a radius, and each wall is described by a line segment 
connecting two points.

\subsection{Initial conditions}
The initial conditions are the following:

\begin{itemize}
    \item A position, $\vec{X_{\alpha}}$ and a desired velocity vector, $\vec{V^{0}_{\alpha}}$.
    \item Two points, $W_n$ for each wall.
\end{itemize}

The walls are set up to form the room we want to simulate, and the desired 
destinations are set outside the exit(s) for all actors. The actors are 
distributed throughout the room and given initial velocities and desired 
walking speed towards the exit by means of some kind of 
distribution\footnote{Further explanation to be added.}.

Also, there is a time step parameter, set initially to $0.1$ seconds. And some 
constants set arbitrarily.
%TODO: Add real description of initial conditions.

We wish to simulate a 10x10m room consisting of four walls and one door, the pedestrians 
are spawned at a random (x,y)-position which can range from $0.30 - 9.70$ for x and y, 
because the radius of the pedestrians is set to $0.30$cm. The pedestrians wants to go 
towards the door, simulating a fire or some other evacuation event. To get a simulation 
up and running, it is nessesary to set up some initial parameters sow: Each pedestrian 
needs to have an initial x,y-position, a desired direction, a desired speed, a maximum 
speed, a potential $V_B$, a radius, a target and the constant A1, A2, B1 and B2 also needs 
to be defined.\\
The desired speeds $v^0$ are Gaussian distributed with mean $<v^0> = 1.34ms^{-1}$ and standard 
diviation $\sqrt{\theta} = 0.26ms^{-1}$.\\
Speeds were limited to $v_\alpha^{max}=1.3 v_\alpha^0$.\\
The potential between pedestrian $\alpha$ and the wall $B$ is given by 

\begin{equation}
V_B=V_B^0 e^{-(r_\alpha - r_B^\alpha )/R} 
\end{equation}
where $V_B^0 = 10m^2s^{-2}$ and $R=0.2m$.
These model parameters have been determined such that they are compatible with empirical data. 
(Kilde: Dirk Helbing and Peter Molnar - Social force model for pedestrian dynamics). This is 
the older article by Helbing, and it seems as if its the same initial conditions, as in 
the article we are working with, except that the force between to pedestrians has changed. 
But still I think we could use the old article to argue why these parameters 
have the given values.\\

\noindent
$A^2_\alpha = 3m/s^2$ and $B^2_\alpha = 0.2 m\\$
$A = 5 m/s^2$ and $B=0.1m$
$r_{\alpha \beta} = 0.6m$
$\lambda_a = 0.75$\\\\
\noindent
T
hese values for the model have been calibrated with empirical data of pedestrian streams.

\subsection{The simulation steps}
The simulation is divided into two parts: Finding the accelerations (or 
resulting force) for all actors, and updating position and velocity for the 
actors.  Since the acceleration for each actor is dependent on both position 
and velocity of the other actors, splitting the calculations this way enables 
us to do the calculations of each actor in any order, and even parallel. The 
drawback is that the actors are only affected by the movement and positions of 
other actors as they were at the end of the last simulation step. This means 
that the time step has to be small enough that this doesn't matter in 
practice.

\subsubsection{Calculating the acceleration vectors}
The acceleration vectors for each actor, $\alpha$, is calculated as follows:

\begin{enumerate}
    \item Determine the acceleration vector provided by the actors' desired 
        direction of movement.
    \item For each other actor $\beta_1\dots\beta_n$, calculate the 
        acceleration vector provided by avoiding the actor $\beta_i$.
    \item Calculate acceleration vectors resulting from avoidance of the walls 
        $w_1\dots w_n$.
    \item Sum all the acceleration vectors to a resultant acceleration vector 
        $a$.
\end{enumerate}

Steps one to three correspond to the three parts of the model.

\subsection{Updating position and velocity}
After every actor has been updated with a resulting acceleration vector from 
the current simulation step, all actors update their position and velocity.  
The position is updated by calculating a displacement vector as follows:

\begin{equation}
    \Delta p = (v_x \Delta t + \frac{1}{2}a_x \Delta t^2, v_y \Delta t + 
    \frac{1}{2}a_y \Delta t^2)
\end{equation}

Where $v_x$ and $v_y$ are the $x$ and $y$ components of the velocity vector, 
$a_x$ and $a_y$ are the $x$ and $y$ components of the acceleration vector, and 
$\Delta t$ is the time step.

After updating the position, the actor's velocity is updated by adding the 
acceleration vector, to be used for the next simulation step.

% start parameters: p_n, v_n for each person.
% p for each wall
%
% steps:
%
% for each person:
%   calculate acceleration from the force parts of the model
%
% for each person:
%   update position by displacement vector
%       delta-p = (v_x*delta-t+1/2*a_x*delta-t^2, 
%       v_y*delta-t+1/2*a_y*delta-t^2)
%   update velocity for next step
%
%
% initial parameters:
%   Set using some sort of distribution around a mean
%
%
% constants:
%   Where do they come from?
