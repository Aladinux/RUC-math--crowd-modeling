% vim:ft=tex
\section{Analysis of the model}
In this section we will go through the model in great detail. First of all, 
as stated briefly earlier this model is an agent based social force model. This means
that the model uses individual entities to say something about the crowd as a whole. 
Each entity or agent is acted on by a collection of different forces. These forces can be
either repulsive or attractive. The general approach of the model is fairly simple. An agent
$\alpha$ wants to go in a desired direction with a desired speed. However the environment 
and other agents might force agent $\alpha$ to stray from the desired direction or have speeds 
different from the desired one. The change in position of agent $\alpha$ is given by:

	\begin{equation}
		\frac{d \vec{r_{\alpha}}}{dt} = \vec{V_{\alpha}} \left( t \right)
	\end{equation}

And, as we know from newtonian physics, the acceleration of agent $\alpha$ is 
then given by the summation of all the forces acting on the agent:

\begin{equation}
    \frac{d \vec{V_{\alpha}}}{dt} = \vec{f_{\alpha}} \left( t \right) + 
    \vec{\xi_{\alpha}}\left( t \right)
\end{equation}

here $\vec{\xi_{\alpha}} \left( t \right)$ is a random fluctuation of agent $\alpha$. This
force is there to incorporate the fact that fact the identical initial condition
generally will not lead to the same series of events. $\vec{f_{\alpha}} \left( t \right)$ 
is a summation of all the forces action in agent $\alpha$ from the environment 
and other agents. More specifically it is given by:

\begin{equation}\label{model}
    \vec{f_{\alpha}} = \vec{f^{0}_{\alpha}}\left( \vec{V_{\alpha}} \right) + 
    \vec{f_{\alpha B}} \left( \vec{r_{\alpha}} \right) +
    \sum_{\beta \neq \alpha} \vec{f_{\alpha \beta}} \left(\vec{r_{\alpha}}, 
    \vec{V_{\alpha}}, \vec{r_{\beta}}, \vec{V_{\beta}} \right) +
    \sum_{i} \vec{f_{\alpha i}} \left( \vec{r_{\alpha}}, \vec{r_{i}}, t 
    \right)
\end{equation}

So this is a summation of four different kinds of forces. We will go through 
them one at a time explaining their role in the model and their mathematical 
structure. The first term on the right hand side is a velocity dependent force 
and it is given by:

\begin{equation}
	\vec{f^{0}_{\alpha}}\left( \vec{V_{\alpha}} \right) =
    \frac{1}{\tau}
    \left( V_{\alpha}^{0} \vec{e_{\alpha}} - \vec{V_{\alpha}} \right)
\end{equation}

Here $\tau$ is the relaxation time, that is the time step between each 
calculation. $\vec{V_{\alpha}}$ is the current velocity of the agent and 
$V_{\alpha}^{0}$ is the initial speed, that is the speed of the agent at the 
end of the last simulation step. $V_{\alpha}^{0}$ is given by:

\begin{equation}
    V_{\alpha}^{0} = \left[ 1 - \eta_{\alpha} \left( t \right) \right] 
    V_{\alpha}^{0} \left( 0 \right) +
    \eta_{\alpha} \left( t \right)V_{\alpha}^{\text{max}}
\end{equation}

Here $V_{\alpha}^{0} \left( 0 \right)$ is the velocity at the beginning of the 
first simulation step and $V_{\alpha}^{\text{max}}$ is the maximum speed that 
the agent can acquire. $\eta_{\alpha}$ is called the impatience of the agent 
and is given by:

\begin{equation}
	\eta_{\alpha} \left( t \right) =
    1 - \frac{\overline{V}_{\alpha} \left( t \right)}
             {V_{\alpha}^{0} \left( t \right)}
\end{equation}

Here $\overline{V}$ is the average speed in the desired direction and as 
earlier $V_{\alpha}^{0} \left( 0 \right)$ is the speed at the beginning of the 
first calculation step of the simulation. Here we can see why this term is 
called the impatience of the agent. If the fraction between the average speed 
in the desired direction and the initial speed is low the impatience term will 
be close to one. When the impatience term is close to one $V_{\alpha}^{0} 
\left( t \right)$ is dominated by $V_{\alpha}^{\text{max}}$. That is, if the 
agent have not moved very far in the desired direction compared to the initial 
speed the impatience of the agent will be close to one. This in term will 
cause the agent's future velocity to be dominated by the maximum velocity of 
the agent.

Now the second term on the right hand side is the forces acting on agent 
$\alpha$ from the walls of the room. The force is given by:

\begin{equation}
    \vec{f_{\alpha B}} \left( \vec{r_{\alpha}} \right) =
    - \nabla_{\vec{r_{\alpha}}} V_{B}
    \left( \| \vec{r_{\alpha}} - \vec{r_{B}^{\alpha}} \| \right)
\end{equation}

Here $\nabla_{\vec{r_{\alpha}}}$ is the gradient and $V_B$ is a repulsive 
potential. $ \| \vec{r_{\alpha}} - \vec{r_{B}^{\alpha}} \|$ is the distance 
from agent $\alpha$ to the nearest point of the nearest wall.

The third term is a summation of all the force between agent $\alpha$ and 
agent $\beta$. It is a function of the position vector and the velocity of 
both agents, and it is given by:

\begin{equation}
    \sum_{\beta \left( \neq \alpha \right)}
        \vec{f_{\alpha \beta }}\left( t \right) =
        A_{\alpha}^{1} exp \left(
            \frac{ r_{\alpha \beta} - d_{\alpha \beta }}
                 {B_{\alpha}^1}
        \right)
    \vec{\eta_{\alpha \beta}} \cdot
    \left(
        \lambda_{\alpha} + \left(
            1 - \lambda_{\alpha}
        \right)
    \right) +
    A_{\alpha}^{2} exp\left(
        \frac{r_{\alpha \beta} - d_{\alpha \beta}}
             {B_{\alpha}^{2}}
    \right)
    \vec{\eta_{\alpha \beta}}
    \label{agentinteraction}
\end{equation}

Here $A_{\alpha}^{1}$, $A_{\alpha}^{2}$, $B_{\alpha}^{1}$, $B_{\alpha}^{2}$ 
and $\lambda_{\alpha}$ are all constants that can differ for each agent. 
$r_{\alpha \beta}$ is the sum of the radii of $\alpha$ and $\beta$ that is 
$r_{\alpha \beta} = r_{\alpha} + r_{\beta}$. $d_{\alpha \beta}$ is the 
distance from the center of mass of agent $\alpha$ and the center of mass of 
agent $\beta$ and is therefore given by $d_{\alpha \beta} = 
\|\vec{X_{\alpha}}\left( t \right) - \vec{X_{\beta}}\left( t \right) \|$. Here 
$\vec{X_{\alpha}}\left( t \right)$ amd $\vec{X_{\beta}}\left( t \right)$ are 
of course the vectors pointing to the center of mass of respectively agent 
$\alpha$ and $\beta$ at the time t. $\eta_{\alpha \beta}$ is the normal vector 
pointing from $\alpha$ to $\beta$ and it is given by:

\begin{equation}
    \eta_{\alpha \beta} =
        \frac{\vec{X_{\alpha}}(t) - \vec{X_{\beta}}(t)}
             {\|\vec{X_{\alpha}}(t) - \vec{X_{\beta}}(t) \|}
\end{equation}

the angle $\phi$ in \eqref{agentinteraction} is the angle between the normal 
vector pointing from agent $\beta$ to $\alpha$ and the direction in which 
agent $\alpha$ is moving. Cosine to the angle is $\cos \left( \phi 
\right)\left( t \right) = - \vec{\eta_{\alpha \beta}}\left( t \right) \cdot 
\vec{e_{\alpha}}\left( t \right)$.

Equation \eqref{agentinteraction} is divided into two terms. The first term on 
the right hand side reflects the agents tendency to stay at a certain distance 
from other agents. This part of the force is called the private sphere because 
the agent prefers to have some free space around him if possible. The radius 
of the private sphere can differ from agent to agent. The constant 
$A_{\alpha}^{1}$, $B_{\alpha}^{1}$ and $\lambda_{\alpha}$ control the nature 
of the private sphere $A_{\alpha}^1$ and $B_{\alpha}^1$ control the strength 
and range of the interaction respectively. $\lambda_{\alpha}$ is there to take 
into account a persons tendency to focus on things happening in front of him 
rather than behind him.

The fourth and last term in \eqref{model} represents the force from attraction 
in the room. Attractions can be either be either interesting sculptures or 
sights or familiar persons the agent prefer to be close to, such as friends 
and family. The mathematical structure of this force is the same as the force 
from other agents, however it is opposite in algebraic sign and has different 
constants. 

\begin{center}
\begin{tabular}{lll}
\hline
\multicolumn{3}{|c|}{\emph{List of constants and variables}}\\
\hline
\small{\textbf{Symbol}} & \small{\textbf{Description}} & \small{\textbf{Unit}}\\
\hline
$A_{\alpha}^{1}$ & \small{Controls the strength of the personal space force}\\
\hline
$A_{\alpha}^{2}$ & \small{Controls the strength of physical collisions}  & \\
\hline
$B_{\alpha}^{1}$ & \small{Controls the range of the personal space force} & \\
\hline
$B_{\alpha}^{2}$ & \small{Controls the range of physical collisions} & \\
\hline
$\lambda_{\alpha}$ &  & \\
\hline
$\vec{f_{\alpha}} \left( t \right)$ & All forces acting on agent $\alpha$  & \\
\hline
$\vec{f_{\alpha B}} \left( \vec{r_{\alpha}} \right)$ & Force on agent $\alpha$ from walls & \\
\hline
$\vec{f_{\alpha \beta}} \left( \vec{r_{\alpha}}, \vec{r_{\beta}}, \vec{V_{\alpha}}, \vec{V_{\beta}} \right)$ & Force on agent $\alpha$ from agent $\beta$ & \\
\hline
$\vec{f_{\alpha i}} \left( \vec{r_{\alpha}}, \vec{r_{i}}, t \right)$ & Force on agent $\alpha$ from attractions & \\
\hline
$V_{\alpha}^{0}$ & Initial speed of agent $\alpha$ & \\
\hline
$V_{\alpha}^{\text{max}}$ & Maximum speed of agent $\alpha$ & \\
\hline
$\vec{V_{\alpha}^{\text{0}}}$ & Desired velocity of agent $\alpha$ & \\
\hline
$\overline{V}_{\alpha}$ & Average speed in desired direction & \\
\hline
$\vec{e}_{\alpha}$ & Vector pointing in desired direction of agent $\alpha$ & \\
\hline
$\vec{r}_{\alpha}\left( t \right) $ & Vector pointing to position of agent $\alpha$ at time t & \\
\hline
$\vec{r}_{B}$ & Vector pointing to nearest point of wall & \\
\hline
$r_{\alpha}$ & Radius of agent $\alpha$ & \\
\hline
$r_{\beta}$ & Radius of agent $\beta$ & \\
\hline
$r_{\alpha \beta}$ & The sum of the radii of agent $\alpha$ and $\beta$ & \\
\hline
$\vec{X}_{\alpha}$ & Vector pointing to center of mass of agent $\alpha$ & \\
\hline
$\vec{X}_{\beta}$ & Vector pointing to center of mass of agent $\beta$ & \\
\hline
$d_{\alpha \beta}$ & Distance between center of mass of agent $\alpha$ and $\beta$ & \\
\hline
$\tau_{\alpha}$ & Relaxation time $\alpha$ and $\beta$ & \\
\hline
$\eta_{\alpha}$ & Impatience of agent $\alpha$ at time t & \\
\hline
$\vec{\eta}_{\alpha \beta}\left( t \right)$ & Normal vector pointing from $\alpha$ to $\beta$ at time t & \\
\hline
$\vec{\xi}\left( t \right)$ & Stochastic element & \\
\hline
$\phi_{\alpha \beta} \left( t \right)$ & Angle between agent $\alpha$ and $\beta$ & \\
\hline
\end{tabular}
\end{center}
