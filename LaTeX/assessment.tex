\section{Assessment of social force models}
% TODO: Add a discussion about the "toy model" thing. Go through the whole 
% section looking at structure, coherency etc.
% Come up with a concept for the section and make it so.
\label{sec:assessment}
When we started to implement and simulate the model we found some fault
in the model, which gave rise to some ideas of how we make the model better.
These minor fault in the model and the solutions to have we could improve it
is described in this section.

\subsection{Limitations of the model}
In this section we discuss a set of features that our models does not have, but
that other social force model have incorporated. These features are a mixture of 
solutions to things that we have ourselves had problems with when working with the 
model and things that we have stumbled across in the literature.

There are several limitations the model that we encountered when we started 
analyse and simulate the model. In this section we discuss some relevant 
limitations of the model that we thought of.

The flexibility of the model in the sence that which situations it could model 
and some limitations of the agents and their awareness of the environment.
E. g. (Comparison article) some people fall in panic situations and become 
obstacles for other people, which is not a feature this model deals with. In 
our model none of the agents fall and become an obstacle force other agents as 
would be seen in real life panic situations.

Another flaw in the model is that the agents do not have way finding, which 
mean that they can not escape a more complex environment with a lot of rooms 
and corridors. As it is now we can only simulate simple cases as the squared 
room and the corridor.  An improvement of the model would then be to implement 
way finding for the agents, so that they can manoeuvre around in other more 
complex cases then the ones we have.

In other models such as the HiDAC, agents can share information about the 
environment, such that the agents have a better idea of where the exit is/are 
(Comparison article). In a real panic situation you would expect people to 
communicate when evacuation a building.

Other models take into account herding behavior between the agents  
\cite{helbing00}, so that some people have tendency to follow other agents, 
while other agents have a tendency to go their own way,  and by doing so 
explore the environment for possible exits.

As the model is now there is no friction between the agents, so that agents 
that have higher velocity than the ones standing in their path, can slide in 
between slower moving agents in front of them, without them slowing down when 
they slide by. In real case scenario people would not slip by that easily 
since the friction between the people would slow them down.  Without friction 
between agents there would be less clogging in front of doors, and the time it 
takes the agents to exit a room would be less than the time expected. Thus the 
validation of the would not be as good as if there would be friction between 
the agents.

At the moment we can not simulate a case where the visibility is very low. The 
agents have a way point to follow, but if the visibility only is 1 meter, e. 
g. in a smoke filled room, the  agents would not have way point to follow, and 
hence round around with out purpose. Here the tangential forces would also 
help the agent to find the exit, but navigation through the tangential force 
from the walls.

Since the model is a two dimensional model it can not handle any motion in the 
vertical dimension. 

\subsection{The state of the social force models}
At this point we have got some understanding of the social force model, and 
also given our opinion about the limitation of the model. We think it is 
reasonable to consider what, if anything, the social force models have taught 
us about crowd behavior. 

The social force model have been used to give both numerical predictions 
and it has verified a series of phenomena observed in reality.

However this kind of knowledge resembles the knowledge you can get from 
the early model of the solar system. These model were quite sophisticated 
such as Potlemy's epicycle model and they were used to correctly predict the 
motion of the planets.

We think that the social force models are of a predictive type rather than 
a explanatory type. They do not give us any knowledge about the underlying 
mechanics of a crowd of pedestrians but they can be used to successfully 
predict quantative behavior of the crowd such as  flow rate. Therefore there 
is nothing wrong with using the model to guide the construction of malls, airports 
and other location where crowds gather.


\subsection{Comments on the development of the social force models}
%-take a step back, discuss the model in general terms
%-is social force a good way to discuss this 
%-incomplete description, inherent in types of paper
While searching for articles leading to social force model, we see there is a 
evolutionary path in transferring physical concepts to describe crowd dynamics.

The success of using fluid dynamics thoughts to model traffic problem encourages 
people to bring the same measurement to model crowd. Later, Helbing improves that 
way by combining the fluid dynamic and gas kinetics concepts \cite{social-force}. 

% However, in order to add the self-organized feature into account, it is necessary 
% to use the agent-based concept.  At the same time, use certain quantity to measure 
% the desire of each agent, and the social force model transfer the "social force" into 
% the calculation in the actual acceleration, which may arise problems because many laws 
% in physics does not apply in this model. 

There is a common feature of the social force model articles, that the method for 
justifying the model is mostly by comparing with observations. It is not 
similar with many cases in classical mechanics, when talking of forces people 
always evaluate energy and momentum, but it is not possible when we have 
the social force involved.
