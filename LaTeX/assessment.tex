\section{Assessment of social force models}
\label{sec:assessment}

% TODO: What is good about social force models.
% TODO: What can we use the model for, provided it works.
% TODO: Now what? perspektivering.
    % Path finding
    % Tangential forces

The social force model we emphasized on in our report is good enough to demonstrate 
the social force concept. In this section we will first introduce some additional features 
, which differ from the ones outlined in section \ref{sec:model-to-simulation} and 
\ref{sec:discussion} in the way that they are not features that are needed to 
replicate the results from the articles, but are extensions to the model.
Later we will point out our major concern about social force model, even if 
such features are inserted. At last, give our opinion about the current state of 
social force models in general and what social force model is good at.

\subsection{Extensions of the model}
% TODO: This section is about minor limitations to our model that have 
% been solved by others.
During the analyses and simulation of the model we have encountered 
several minor limitations. In this section we discuss these limitations, 
some of which have been solved by others in the field. 

The variety of situations the model can handle is greatly affected by the 
pedestrians' ability to manoeuvre around in the environment. 
Features such as 
\emph{path finding} and \emph{lowered visibility} allowed the model to design 
more complicated situation, other features such as \emph{communication 
between pedestrians}, \emph{falling} and \emph{herding behaviour} can represent 
more thorough psychological state. We will discuss these features one at a time.

\begin{itemize}
\item {Path finding: } 
In our model the pedestrians do not have path finding, which means that the model 
can not simulate people escaping a complex environment with a lot of rooms 
and corridors. A example of a social force model with pathfinding is the HiDAC 
model\cite{HiDAC}.

\item {Lowered visibility: }
At the moment we can not simulate a situations where the visibility is very low 
e.g. a room filled with smoke. In the model we have worked with The pedestrians always 
have a waypoint to steer towards, but if the visibility only is 1 meter the 
pedestrians can not necessarily see the waypoint \cite{HelbingNew}. This could be 
implemented by adding a tangential component to the force coming from the wall. 
Sow people will follow the wall when they get close to it.

\item {Communication between pedestrians: }
In real panic situations it is not hard to imagine people communicating to help each 
other finding the exits or using alternative routes. This is 
not something that our model is capable of. For a model with communication between 
pedestrians see \cite{HiDAC}.

\item {Falling and injuries }
In situations of crowd panic, people are often injured. This is either due to the 
enormous pressure that can built up in a crowd or due to people falling and being stamped. Our 
does not take into account these possibilities. In the HiDAC model people can 
be injured and even fall and become obstacles for other pedestrians.\cite{HiDAC}

\item {Herding}
Other models take into account herding behaviour between the pedestrians 
\cite{helbing00}, so that some people have tendency to follow other pedestrians, 
while other pedestrians have a tendency to go their own way,  and by doing so 
explore the environment for possible exits.

\end{itemize}

These features all add up to give a more realistic simulation of crowds, and 
obviously there are many more features than those that people can think of 
and could be solved. 
Next we will refer to the governing assumption behind the social force
model, where comes our biggest concern.

\subsection{Inserting social force concept into classical mechanics}
\label{subsec:development}
% This section is about a major limitation/error in the model that 
% cannot be solved easily.
% Discussion about using physological concepts in the model.

Previously we have seen some features that could improve the 
predictions made by the model. However, there is a major problem that lies 
in the model and can not be fixed by adding a new feature, when the social 
force is added into the way of classical mechanic calculation.

In the early research for pedestrian behaviour, people try to apply the existing 
models in the physical world to describe a crowd, but there is no term describing 
a molecule's will in models like those. Therefore, in order to make the model 
human, Helbing gets the revolutionary idea of measuring the psychological state 
by a term called "social force", which is not a real force in the physical world 
\cite{social-force}, but a representation of a desire to reach a given position. 
Then the measurements of social and physical features of pedestrians (those 
measurements are all put into the catalogue called "force") are gathered together 
and calculated within the limit of classical mechanics, which is attacked by many 
laws in the physical world, for example, Newton's third law and the conservation of 
energy. Then this model is not a model in physics as it does not fit nicely in the 
whole physical world. However, it insists in describing the physical behaviour, 
because what we get directly from the model are the positions and velocities of 
pedestrians. In general, the model grabs some ingredients from both social and 
physical aspects, put them into the physics machine, and comes up with the 
product that falls in the physics catalogue. In that manner, the model is not 
convincing to us.

The combination of actual physics and measure of psychology state gives 
rise to some problems in application. For example, in order to measure the pressure 
in a crowd simulated by such a model, it is not clear how to cut out the physical force 
when the interactions between pedestrians are partly representing desires. 

Besides, the inconsistency gives a challenging work to justify the model in general, 
because there is no governing theory that can do the job. The only way from Helbing 
is by comparing the outcome of the model to the reality. In order to achieve 
the fitness, many parameters need to be adjusted at the same time, and we do 
not even know how to tune the parameters precisely, which makes the way to 
get the fitness tedious. Moreover, if the model is used to make predictions, the
outcomes are not persuasive . 

Therefore, the model is not qualified in making precise prediction in the classic 
manner. however, it does give us some valuable idea about crowd behaviour, which 
will be discussed next.

\subsection{Comments on the state of the social force models}
% This is sort of a conclusion on the last two sections. What have we 
% learned about crowds from the modelling. What is the status of this 
% kind of model. Talk about the fact that it seems that we need to 
% change the parameters for each case we want to simulate.
At this point we have got some understanding of the social force model, and 
also given our opinion about the limitation of the model. We think it is 
reasonable to consider what, if anything, the social force models have taught 
us about crowd behaviour and comment on the state of social force model. 

The social force model have been used to give both numerical predictions of 
real life situations and it has been able to construct a series of phenomena 
observed in reality such as lane formation and freezing by heating.
However this kind of knowledge resembles the knowledge people can get from 
the early model of the solar system. Some these model were quite sophisticated 
such as Potlemy's epicycle model and they were used to correctly predict the 
motion of the planets and describe (although incorrectly) phenomena such as 
solar eclipses.
We think that the social force models are in the same state as those models, 
they are of a predictive type rather than a explanatory type. They do not give 
us any knowledge about the underlying mechanics of a crowd of pedestrians but 
they can be used to calculate quantitative behaviour of the crowd such 
as flow rate. This information can be used to guide the construction of malls, 
airports and other location where crowds gather.

In the article \cite{self-org}  we get the following comment:

\begin{quote}
	The advantage of the social-force-based simulation
	approach is its simple form and its small number of
	parameters, which do not need to be calibrated anew
	for each situation.
\end{quote}

However in our case we have had to calibrate the parameters for each 
situation, for example, if we keep the same values of the parameters when 
change the scenario from square room to the corridor, there might appear 
absurd phenomena such as pedestrians walk through the wall fearlessly. 

We have mostly in this section talked about the weakness of social force 
model, but what do we learn from the model? This model is innovative in 
measuring human desire quantitatively by simply taking the parameters for motion. 
A human desire to get from A to B is a very abstract quantity, and a ton of
different circumstances could effect the exact movement of this
individual. It has been shown that people can take this very abstract quantity 
calculated in the simple manner and transform it into something very 
concrete with surprising reasonable results.
Through simulations we can get an idea of the motion of a crowd instead of an 
individual, while if we keep track of one pedestrian he sometimes behaves strangely. 
Therefore, the model emphasize on the behaviour of a crowd in general even though 
it starts from describing the motion of a pedestrian. Through experiments 
it has been shown that such an interpretation can make some rather realistic 
predictions about how the crowed moves. We may not trust some particular result from 
the model, but the simulation can be a reference if somebody is required to estimate 
crowd behaviour in a particular situation.

% perspective
Since the simulation we present in the report has not shown some properties of a 
crowd, if given more time our group would like try to solve the discrepancies and add the 
possible solutions that has been talked about in section \ref{sec:discussion}. The first thing to do 
maybe modify the repulsive forces and add the frictional force, and make them 
velocity dependent, which will make the crowd behave more realistic. 
To enable the model to simulate more complex situation, we will need the path 
finding feature. Also a set of parameters should be determined when the environment is 
changed, and the method to attain those parameters may be the same as what the 
Helbing group has been doing, that is by doing some experiments and analysing the 
 video track.