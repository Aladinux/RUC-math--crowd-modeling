\section{Assessment of social force models}
\label{sec:assessment}
In this section we will present our assessment of social force models in 
general, based on our work with the specific model and the results we have 
obtained. To expand the scope of the discussion, we will first outline various 
extensions to social force models that are presented elsewhere, which 
introduce some features into the models that are lacking in the version we 
have worked with. Based on our results and these extensions, we will then 
outline the advantages and weaknesses of social force models in general, and 
conclude by assessing the state of these types of models.

\subsection{Extensions of the model}
There are several features that might be added to the social force model we 
have worked with, that will widen the scope of scenarios for which the model 
is useful. As mentioned in section~\ref{sec:variants}, there already exists 
models that incorporate some of these features. In this section we will 
outline some of these features, go give a broader view of what is possible 
with social force-based models.

The features we will look at are \emph{path finding}, which will allow the 
model to simulate more complicated scenarios, and \emph{communication between 
pedestrians}, \emph{falling} and \emph{handling lowered visibility} that can 
add types of pedestrian behaviour that is not handled well in our version of 
the model.

\begin{itemize}
    \item \textbf{Path finding:} Our model does not incorporate path 
        finding, i.e. pedestrians only move towards one fixed target. Adding 
        path finding would allow the model to simulate people moving in more 
        complex environments than what is possible with our model. There 
        exists various models that incorporate this feature \cite{HiDAC}.
 
    \item \textbf{Communication between pedestrians:} Adding an ability for 
        pedestrians to communicate would enable us to make simulations of 
        cases where groups of pedestrians follow each other (such as a 
        family), and it would enable pedestrians to cooperate in finding 
        alternative routes. Models adding communication are also seen in the 
        literature \cite{HiDAC}.

    \item \textbf{Falling and injuries:} In situations of crowd panic, people 
        are often injured. This is either due to the enormous pressure that 
        can built up in a crowd or due to people falling and being trampled.  
        Our does not take into account these possibilities, but other models 
        do \cite{HiDAC}.

    \item \textbf{Lowered visibility:} Our model assumes that pedestrians can 
        identify their target and move towards it. This means that we cannot 
        simulate scenarios where pedestrians do not know in which direction 
        they want to move, e.g. a room filled with smoke. Adding alternative 
        mechanisms for pedestrian movement, such as having pedestrians close 
        to walls moving along them, would allow us to simulate pedestrian 
        movement in cases of lowered visibility \cite{HelbingNew}.
\end{itemize}

These features allows the social force models to simulate more complex cases 
than what is possible using our model, and since they have been incorporated 
into other models, we will include them in our assessment of the pros and cons 
of social force models.

\subsection{Advantages of social force models}
\subsection{Weaknesses of social force models}

\subsection{Inserting social force concept into classical mechanics}
\label{subsec:development}
% This section is about a major limitation/error in the model that 
% cannot be solved easily.
% Discussion about using physological concepts in the model.

Previously we have seen some features that could improve the 
predictions made by the model. However, there is a major problem that lies 
in the model and can not be fixed by adding a new feature, when the social 
force is added into the way of classical mechanic calculation.

In the early research for pedestrian behaviour, people try to apply the existing 
models in the physical world to describe a crowd, but there is no term describing 
a molecule's will in models like those. Therefore, in order to make the model 
human, Helbing gets the revolutionary idea of measuring the psychological state 
by a term called "social force", which is not a real force in the physical world 
\cite{social-force}, but a representation of a desire to reach a given position. 
Then the measurements of social and physical features of pedestrians (those 
measurements are all put into the catalogue called "force") are gathered together 
and calculated within the limit of classical mechanics, which is attacked by many 
laws in the physical world, for example, Newton's third law and the conservation of 
energy. Then this model is not a model in physics as it does not fit nicely in the 
whole physical world. However, it insists in describing the physical behaviour, 
because what we get directly from the model are the positions and velocities of 
pedestrians. In general, the model grabs some ingredients from both social and 
physical aspects, put them into the physics machine, and comes up with the 
product that falls in the physics catalogue. In that manner, the model is not 
convincing to us.

The combination of actual physics and measure of psychology state gives 
rise to some problems in application. For example, in order to measure the pressure 
in a crowd simulated by such a model, it is not clear how to cut out the physical force 
when the interactions between pedestrians are partly representing desires. 

Besides, the inconsistency gives a challenging work to justify the model in general, 
because there is no governing theory that can do the job. The only way from Helbing 
is by comparing the outcome of the model to the reality. In order to achieve 
the fitness, many parameters need to be adjusted at the same time, and we do 
not even know how to tune the parameters precisely, which makes the way to 
get the fitness tedious. Moreover, if the model is used to make predictions, the
outcomes are not persuasive . 

Therefore, the model is not qualified in making precise prediction in the classic 
manner. however, it does give us some valuable idea about crowd behaviour, which 
will be discussed next.

\subsection{Comments on the state of the social force models}
% This is sort of a conclusion on the last two sections. What have we 
% learned about crowds from the modelling. What is the status of this 
% kind of model. Talk about the fact that it seems that we need to 
% change the parameters for each case we want to simulate.
At this point we have got some understanding of the social force model, and 
also given our opinion about the limitation of the model. We think it is 
reasonable to consider what, if anything, the social force models have taught 
us about crowd behaviour and comment on the state of social force model. 

The social force model have been used to give both numerical predictions of 
real life situations and it has been able to construct a series of phenomena 
observed in reality such as lane formation and freezing by heating.
However this kind of knowledge resembles the knowledge people can get from 
the early model of the solar system. Some these model were quite sophisticated 
such as Potlemy's epicycle model and they were used to correctly predict the 
motion of the planets and describe (although incorrectly) phenomena such as 
solar eclipses.
We think that the social force models are in the same state as those models, 
they are of a predictive type rather than a explanatory type. They do not give 
us any knowledge about the underlying mechanics of a crowd of pedestrians but 
they can be used to calculate quantitative behaviour of the crowd such 
as flow rate. This information can be used to guide the construction of malls, 
airports and other location where crowds gather.

In the article \cite{self-org}  we get the following comment:

\begin{quote}
	The advantage of the social-force-based simulation
	approach is its simple form and its small number of
	parameters, which do not need to be calibrated anew
	for each situation.
\end{quote}

However in our case we have had to calibrate the parameters for each 
situation, for example, if we keep the same values of the parameters when 
change the scenario from square room to the corridor, there might appear 
absurd phenomena such as pedestrians walk through the wall fearlessly. 

We have mostly in this section talked about the weakness of social force 
model, but what do we learn from the model? This model is innovative in 
measuring human desire quantitatively by simply taking the parameters for motion. 
A human desire to get from A to B is a very abstract quantity, and a ton of
different circumstances could effect the exact movement of this
individual. It has been shown that people can take this very abstract quantity 
calculated in the simple manner and transform it into something very 
concrete with surprising reasonable results.
Through simulations we can get an idea of the motion of a crowd instead of an 
individual, while if we keep track of one pedestrian he sometimes behaves strangely. 
Therefore, the model emphasize on the behaviour of a crowd in general even though 
it starts from describing the motion of a pedestrian. Through experiments 
it has been shown that such an interpretation can make some rather realistic 
predictions about how the crowed moves. We may not trust some particular result from 
the model, but the simulation can be a reference if somebody is required to estimate 
crowd behaviour in a particular situation.
