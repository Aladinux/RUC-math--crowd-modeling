\section{Assessment of social force models}
\label{sec:assessment}
In this section we will present our assessment of social force models in 
general, based on our work with the specific model and the results we have 
obtained. Based on our results, we will outline the strengths and weaknesses 
of social force models in general, and conclude by assessing how well these 
types of models simulate crowd behaviour.

\subsection{Strengths of social force models}
The main advantage of using social force models for modelling crowds is that 
it allows us to simulate a complex system using a quite simple formulation of 
the model. Simulating a large crowd of pedestrians using a classical (i.e. 
non-agent based) model would be very complex, and quite possibly impractical.

This simplification is possible because the complexities of the model 
behaviour is moved from the formulation of the model and into the 
calculations. That is, a very large number of calculations are needed to get 
any meaningful results out of this model. This means that working with this 
kind of model would be impractical without the help of computers, and indeed a 
large quantity of processing power is necessary to get any results. In our 
simulations this has been most apparent in the need to implement the 
calculation-intensive parts of the model in the C programming language, to be 
able to achieve reasonable computation times.

Another advantage of the simulations we get from the model, is that it is very 
straight forward to inspect the results visually, because it is possible to 
create drawings of the simulation steps. This means that comparing the 
simulations to e.g. videos of real-life crowds, and spotting effects such as 
the lane formation becomes trivial. It is also an advantage that simulations 
can be run in real time, so making visual assessments of results is easy.

Of course having a simple model that is practical to implement is of no use if 
it does not give useful results. Empirical studies have shown that social 
force models are able to show phenomena that correspond to real crowds, and as 
such do provide meaningful results in some cases \cite{self-org,HelbingNew}. 

\subsection{Weaknesses of social force models}
While social force models in some cases have been shown to give results that 
correspond to real life observations, there are several weaknesses to this 
approach to modelling crowds. Some of these weaknesses are related to the way 
the models are presented, and some are more fundamental to the nature of 
social force models.

When working with social force models, we have had to piece together a working 
model from several different sources. This exposes a difficulty in assessing 
the models: It is not always obvious if a given weakness is due to an inherent 
quality of the models, or if it is simply due to a weak or missing formulation 
of some part of it. Especially precise results of simulations have been 
difficult to find, and as has been shown in 
section~\ref{sec:model-to-simulation}, we have had to fill in several details 
ourselves. As mentioned above, the difficulty in estimating parameters has 
also provided a barrier in this respect. In addition, as mentioned in 
section~\ref{sec:varying-constants}, it is difficult to assess whether a 
simulation run provides meaningful results, so determining which parameter 
values work further adds to the difficulty of evaluating the model.

Setting aside the difficulties in finding detailed information about the 
model, it is readily apparent that social force models are in a relatively 
early state of development, so the amount of empirical data available to 
assess the quality of the models' predictions is quite low. This means that 
even if the social force models have shown some promising results, it is 
impossible to say with confidence that they do indeed predict the actual 
behaviour of crowds very well.

Using the distinction between simulation-based models and theoretical models 
laid out in \cite{imfufa-tekst}, it is clear that social force models belong 
in the category of simulation-based models. That is, even though the notation 
and concepts are borrowed from physics, there is no underlying theory behind 
the formulation of social force models. Instead, it is a model that is 
formulated to simulate a concrete set of observations of crowd behaviour. This 
means that in order to build confidence in the model's ability to provide 
useful results, it has to be shown to conform to empirical data with a high 
confidence. Something that these models have not achieved, at least so far.

Further compounding these uncertainties is the fact that the model is 
formulated in a way that is counter-intuitive to the way human behaviour is 
normally perceived. One of the reasons we are sceptic that these forces are 
able to completely explain human behaviour, is that they pertain to behaviour 
of objects in a physical world, but they do not obey the traditional physical 
laws of motion. Additionally, that human behaviour is reducible to simple 
repulsive forces is counter to what we believe is reasonable. This means that 
if this is indeed the case, strong evidence is needed to convince us, which is 
not currently provided.   

Finally, further cementing the social force models' status as simulation type 
models, is the fact that different variants of social force models use 
completely different parameters and formulations of the different forces, 
sometimes even contradicting each other, making it apparent that the models 
are changed in arbitrary ways to better match empirical observations. This 
belief is corroborated by the fact that the models do not make any new 
predictions that are then tested, but instead only seem to attempt to 
replicate already observed behaviour.

\subsection{Conclusion on the assessment}
Weighing the advantages and disadvantages of social force models against each 
other, it is quite apparent that the models do not instil a strong sense of 
confidence in their predictions. However, the crucial advantage that these 
models have, is that they in some cases are able to provide reasonable 
simulations of crowd behaviour, that no other models (that we have seen) have 
been able to. This means that while social force models are far from perfect, 
they are in many ways the best available way of evaluating e.g. a new 
building's suitability for efficient crowd movement. And while the models may 
not be able to provide any underlying theory or reason for the crowd's 
behaviour, in practice they may, given further adjustments and experiments 
with real life observations, be a substantial improvement over today's 
standards.
