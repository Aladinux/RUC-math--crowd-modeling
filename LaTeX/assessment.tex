\section{Assessment of social force models}
\label{sec:assessment}
In this section we first discuss the limitations of the social force models 
that we have worked with. These limitation differ from the ones outlined in 
section \ref{sec:lack} and \ref{sec:discrepancies} in the way that 
they are not features that are needed to replecate the results from the article, 
but are extensions to the model. 
While going through the literature of the field we have noticed a trend in the 
publications and this will be discussed in section \ref{subsec:development}. 
We then move on to comment on the status of social force models in general.

\subsection{Limitation and extensions of the model}
There are several limitations the model that we encountered when we started 
analyse and simulate the model. In this section we discuss some relevant 
limitations of the model, some of which have been solved by other people in the 
field. 

The variety of situations the model can handle is greatly decreased by the 
agents ability to maneuver around in the environment. The maneuverability of 
the agents can be increased by incoorporating a series of features such as 
\emph{path finding} and \emph{communication between agents}. 

In real panic situations one could expect to see a series of events that our 
model is incapable of simulating. Such as people \emph{falling} or showing a 
\emph{herding behavior}.

\subsubsection{Path finding}
In our model the agents do not have path finding, which means that the model 
can not simulate a escapes of complex environment with a lot of rooms 
and corridors. A example of a social force model with pathfinding is the HiDAC 
model \cite{HiDAC}.

\subsubsection{Communication between agents}
In real panic situations it is not hard to imagine people communication with 
each other to help each other find the exits or use alternative routes. This is 
not something that our model is capable of. For a model with communication between 
agents see \cite{HiDAC}.

\subsubsection{Lowered visibility}
At the moment we can not simulate a case where the visibility is very low. The 
agents have a way point to follow, but if the visibility only is 1 meter, e. 
g. in a smoke filled room, the  agents would not have way point to follow, and 
hence round around with out purpose. Here the tangential forces would also 
help the agent to find the exit, but navigation through the tangential force 
from the walls. %find reference

\subsubsection{Falling}
In \cite{HiDAC} people have the ability fall in panic situations and become 
obstacles for other people, which is not a feature this model deals with. In 
our model none of the agents fall and become an obstacles for other agents as 
would be seen in real life panic situations.

\subsubsection{Herding}
Other models take into account herding behavior between the agents  
\cite{helbing00}, so that some people have tendency to follow other agents, 
while other agents have a tendency to go their own way,  and by doing so 
explore the environment for possible exits.

\subsection{Comments on the development of the social force models}
\label{subsec:development}
%-incomplete description, inherent in types of paper
While searching for articles leading to social force model, we see there is a 
evolutionary path in transferring physical concepts to describe crowd dynamics.

The success of using fluid dynamics to model traffic problem encourages 
people to bring the same approach to model crowds. Helbing later improved the  
model by combining fluid dynamic and gas kinetics concepts\cite{social-force}. 

However, in order to add the self-organized feature into account, it is necessary 
to use the agent-based concept.  At the same time, use certain quantity to measure
the desire of each agent, and the social force model transfer the "social force" into 
the calculation in the actual acceleration, which may arise problems because many laws 
in physics does not apply in this model. 

This approach of building on earlier models is apparent when reading through the 
literature. One article does rarely contain all the information needed to replicate 
the results given in the article.

\subsection{Comments on the state of the social force models}
%-take a step back, discuss the model in general terms
%-is social force a good way to discuss this
% This stuff in this discussion should hold even if the model had all the features listed in the first subsection.
At this point we have got some understanding of the social force model, and 
also given our opinion about the limitation of the model. We think it is 
reasonable to consider what, if anything, the social force models have taught 
us about crowd behavior. 

The social force model have been used to give both numerical predictions 
and it has verified a series of phenomena observed in reality such as 
lane formation and freezing by heating.

However this kind of knowledge resembles the knowledge you can get from 
the early model of the solar system. These model were quite sophisticated 
such as Potlemy's epicycle model and they were used to correctly predict the 
motion of the planets and describe (although incorrectly) phenomena such as 
solar eclipses.

We think that the social force models are of a predictive type rather than 
a explanatory type. They do not give us any knowledge about the underlying 
mechanics of a crowd of pedestrians but they can be used to successfully 
predict quantative behavior of the crowd such as  flow rate. Therefore there 
is nothing wrong with using the model to guide the construction of malls, airports 
and other location where crowds gather.