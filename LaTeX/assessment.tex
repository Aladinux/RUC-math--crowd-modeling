\section{Assessment of social force models}
\label{sec:assessment}
In this section we first discuss the limitations of the social force models 
that we have worked with. These limitation differ from the ones outlined in 
section \ref{sec:lack} and \ref{sec:discrepancies} in the way that 
they are not features that are needed to replicate the results from the articles, 
but are extensions to the model.

\subsection{Extensions of the model}
% TODO: This section is about minor limitations to our model that have 
% been solved by others.
There are several limitations the model that we encountered when we started 
analyse and simulate the model. In this section we discuss some relevant 
limitations of the model, some of which have been solved by other people in the 
field. 

The variety of situations the model can handle is greatly decreased by the 
pedestrians ability to maneuver around in the environment. The maneuverability of 
the pedestrians can be increased by incoorporating a series of features such as 
\emph{path finding} and \emph{communication between pedestrians}. 

In real panic situations one could expect to see a series of events that our 
model is incapable of simulating. Events such as people \emph{falling} or showing a 
\emph{herding behavior}. We will discuss these features one at a time.

\subsubsection{Path finding}
In our model the pedestrians do not have path finding, which means that the model 
can not simulate people escaping a complex environment with a lot of rooms 
and corridors. A example of a social force model with pathfinding is the HiDAC 
model\cite{HiDAC}.

\subsubsection{Communication between pedestrians}
In real panic situations it is not hard to imagine people communication with 
each other to help each other find the exits or use alternative routes. This is 
not something that our model is capable of. For a model with communication between 
pedestrians see \cite{HiDAC}.

\subsubsection{Lowered visibility}
At the moment we can not simulate a situations where the visibility is very low 
e.g. a room filled with smoke. In the model we have worked with The pedestrians always 
have a waypoint to steer towards, but if the visibility only is 1 meter the 
pedestrians can not necessarily see the waypoint. \cite{HelbingNew}

\subsubsection{Falling and injuries}
In situations of crowd panic people are often injured. This is either due to the 
enormous pressure that can built up in the crowd or due to people falling. Our 
does not take into account these possibilities. In the HiDAC model people can 
be injured and even fall and become obstacles for other pedestrians.\cite{HiDAC}

\subsubsection{Herding}
Other models take into account herding behavior between the pedestrians  
\cite{helbing00}, so that some people have tendency to follow other pedestrians, 
while other pedestrians have a tendency to go their own way,  and by doing so 
explore the environment for possible exits.

These features all add up to give a more realistic simulation of crowds 
but do simulation like these teach us anything about why a crowd behaves 
as it does?

\subsection{Inserting social force concept into classical mechanics}
\label{subsec:development}
% This section is about a major limitation/error in the model that 
% cannot be solved easily.
% Discussion about using physological concepts in the model.
While searching for articles leading to social force model, we see there is a 
evolutionary path in transferring physical concepts to describe crowd dynamics.

The success of using fluid dynamics to model traffic problem encourages 
people to bring the same approach to model crowds. Helbing later improved the  
model by combining fluid dynamic and gas kinetics concepts\cite{social-force}. 

However, in order to add the self-organized feature into account, it is necessary 
to use the agent-based concept.  At the same time, use certain quantity to measure
the desire of each agent, and the social force model transfer the "social force" into 
the calculation in the actual acceleration, which may arise problems because many laws 
in physics does not apply in this model as the social force is not a real 
physical force.

\subsection{Comments on the state of the social force models}
% This is sort of a conclusion on the last two sections. What have we 
% learned about crowds from the modelling. What is the status of this 
% kind of model. Talk about the fact that it seems that we need to 
% change the parameters for each case we want to simulate.
At this point we have got some understanding of the social force model, and 
also given our opinion about the limitation of the model. We think it is 
reasonable to consider what, if anything, the social force models have taught 
us about crowd behavior. 

The social force model have been used to give both numerical predictions 
and it has verified a series of phenomena observed in reality such as 
lane formation and freezing by heating.

However this kind of knowledge resembles the knowledge you can get from 
the early model of the solar system. These model were quite sophisticated 
such as Potlemy's epicycle model and they were used to correctly predict the 
motion of the planets and describe (although incorrectly) phenomena such as 
solar eclipses.

We think that the social force models are of a predictive type rather than 
a explanatory type. They do not give us any knowledge about the underlying 
mechanics of a crowd of pedestrians but they can be used to successfully 
predict quantative behavior of the crowd such as flow rate. Therefore there 
is nothing wrong with using the model to guide the construction of malls, airports 
and other location where crowds gather.
