\documentclass[10pt,a4paper]{article}
\usepackage[latin1]{inputenc}
\usepackage{amsmath}
\usepackage{amsfonts}
\usepackage{amssymb}
\begin{document}


\subsection*{Linear approximation}
In our simulation, we use Euler's method to approximate solutions to the equations of motion, which are ordinary differential equations with initial conditions calculated from last time step, and those initial value problems look like:

\begin{eqnarray}
\frac{dy}{dt} &=& f(y),\\
y(t_{0}) &=& y_{0}
\end{eqnarray}

Within some small time interval from $t_{0}$ to $t_{0}+\bigtriangleup t$, the function of $ y(t) $ can be approximately represented by a straight line:

\begin{equation}
l(t) = y_{0} + f(y_{0}) (t-t_{0})
\end{equation}

then after a small step size, the function value $ y $ becomes
\begin{equation}
y = y_{0} + f(y_{0}) \bigtriangleup t
\end{equation}
and the calculated value of $ y $ also becomes the initial value for the following step. However, this kind of method can cause error in most situations, and there are methods trying to minimize the error caused by the numerical way of solving problem.

\subsection*{Step size}
We can see that the error caused by the numerical approach must depends on the step size.  From the mathematical point of view, a smaller step size generally gives smaller error, but as we use the computer to do the calculation and it only deals with rational numbers, so the round off error becomes prominent. Therefore, an optimal step size could be the one that causes the sum of two sources of errors smallest.

\end{document}


