\documentclass[10pt,a4paper]{article}
\usepackage[latin1]{inputenc}
\usepackage{amsmath}
\usepackage{amsfonts}
\usepackage{amssymb}
\begin{document}
\section{Description of initial conditions}
We wish to simulate a 10x10m room consisting of four walls and one door, the pedestrians are spawned at a random (x,y)-position which can range from $0.30 - 9.70$ for x and y. The pedestrians wants to go towards the door, simulating a fire or some other evacuation event. To get a simulation up and running, it is nessesary to set up some initial parameters sow: Each pedestrian needs to have an initial x,y position, a desired direction, a desired speed, a maximum speed, a potential $V_B$, a radius, a target and the constant A1, A2, B1 and B2 also needs to be defined.\\\\
\noindent
The desired speeds $v^0$ are Gaussian distributed with mean $<v^0> = 1.34ms^{-1}$ and standard diviation $\sqrt{\theta} = 0.26ms^{-1}$. \\\\
Speeds were limited to $v_\alpha^{max}=1.3 v_\alpha^0$.\\\\
\noindent
The potential between pedestrian $\alpha$ and the wall $B$ is given by $V_B=V_B^0 e^{-(r_\alpha - r_B^\alpha )/R}$ where $V_B^0 = 10m^2s^{-2}$ and $R=0.2m$\\\\
\noindent
These model parameters have been determined such that they are compatible with empirical data. (Kilde: Dirk Helbing and Peter Molnar - Social force model for pedestrian dynamics). This is the older article by Helbing, and it seems as if its the same initial conditions, as in the article we are working with, except that the force between to pedestrians has changed. But still I think we could use the old article to argue why these parameters have the given values.\\\\
\noindent
$A^2_\alpha = 3m/s^2$ and $B^2_\alpha = 0.2 m\\$
$A = 5 m/s^2$ and $B=0.1m$
$r_{\alpha \beta} = 0.6m$
$\lambda_a = 0.75$\\\\
\noindent
These values for the model have been calibrated with empirical data of pedestrian streams.
\end{document}