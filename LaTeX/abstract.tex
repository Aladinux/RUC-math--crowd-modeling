% vim:ft=tex

\begin{abstract}
    \section*{Abstract}
    \small
    We look at social force models as a way to model the behaviour of human 
    crowds, in order to evaluate what these types of models can tell us about 
    crowds, which assumptions underlie them and what their strengths and 
    weaknesses are. In order to do this evaluation, we implement a computer 
    simulation of an exemplary social force model.

    In order to create this simulation, we pick an exemplary model that is 
    well described in the article that presents it, and analyse it in detail, 
    filling in details from other articles when necessary. Based on this 
    analysis of the model, we go from the abstract model formulation to a 
    concrete numerical simulation by filling in required details, such as  how 
    to approximate the movement of pedestrians, how to set initial conditions 
    and values, and how to implement the interaction between pedestrians and 
    walls in practice.

    From our results, it is clear that our simulation (with the right 
    parameters) exhibits reasonable pedestrian behaviour upon visual 
    inspection. While we successfully replicate some results from the 
    literature, other effects do not manifest themselves. We discuss several 
    reasons for this discrepancy, including features that are missing from the 
    model, parameter values, effects of using random numbers to generate the 
    initial conditions and possible errors in our implementation of the model.

    Based on the results of our own simulations and our review of the social 
    force modelling field, we assess social force models and their strengths 
    and weaknesses. We conclude that social force models are not based on any 
    theories for the behaviour of crowds, but are created to replicate a set 
    of observations. As such, any confidence in their predictions must come 
    from a record of producing results fitting observations; and since the 
    field is relatively new, they have not yet reached this state. Social 
    force models do, however, provide a practical way to simulate something 
    that would otherwise be impossible to simulate. As such, they are the best 
    available way to provide e.g. guidance when designing facilities that must 
    accommodate many pedestrians, and given time the accuracy of their 
    predictions will probably increase.
\end{abstract}
\begin{abstract}
    \section*{Resume}
    \small
    Vi ser på ``social force''-modeller som et værktøj til at modellere 
    menneskemængders opførsel, for at vurdere hvad disse typer af modeller kan 
    fortælle os om menneskemængder, hvilke antagelser ligger til grund for 
    modellerne, og hvad deres styrker og svagheder er. For at udføre denne 
    vurdering, udarbejder vi en computersimulering af en eksemplarisk social 
    force-model.

    For at udarbejde denne simulering, udvælger vi en eksemplarisk model som 
    er velbeskrevet i den artikel der præsenterer den, og analyserer den i 
    detaljer. Herunder udfylder vi hvor det er nødvendigt huller i 
    modelformuleringen vha. dele fra andre artikler. Baseret på denne analyse 
    af modellen, går vi fra det abstrakte formulering af modellen til en 
    konkret nummerisk simulering ved at udfylde nødvendige detaljer, såsom 
    hvordan vi tilnærmer fodgængernes bevægelse, hvordan vi opstiller 
    begyndelsesbetingelser, og hvordan vi udregner samspillet mellem 
    fodgængere og vægge i praksis.

    Fra vores resultater er det klart, at vores simulering (med de rette 
    parametre) udviser rimelig fodgængeradfærd vurderet ved visuel  
    inspektion. Mens det lykkes os at genskabe nogle af resultaterne fra 
    litteraturen, er der andre effekter der ikke optræder. Vi diskuterer en 
    række  årsager til denne forskel, herunder manglende features i modellen, 
    indstilling af parametre, effekter af at anvende tilfældige tal til at 
    generere begyndelsesbetingelser samt eventuelle fejl i vores implementering  
    af modellen.

    Baseret på resultaterne af vores egne simuleringer og vores gennemgang af 
    forskningsområdet, vurderer vi social force-modeller og deres styrker og 
    svagheder. Vi konkluderer at social force-modeller ikke bygger på nogen 
    teori for menneskemængders adfærd, men er skabt for at lave simuleringer 
    svarende til konkrete observationer. Som sådan skal tilliden til deres 
    forudsigelser opbygges ved at sammenholde simuleringer med observationer, 
    og da feltet er relativt nyt, har modellerne endnu ikke opnået stor 
    sikkerhed i deres resultater.  Social force-modeller giver dog en praktisk 
    måde at simulere noget som ellers ville være umuligt at simulere.  Som 
    sådan, repræsenterer de det bedste værktøj der eksisterer til fx at 
    vejlede ved udformningen af faciliteter der skal rumme mange fodgængere, 
    og pålideligheden af resultaterne vil formentlig forbedres med tiden.
\end{abstract}
