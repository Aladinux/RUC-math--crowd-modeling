% vim:ft=tex

\begin{abstract}
    \section*{Abstract}
    \small
    We look at social force models as a way to model the behaviour of human 
    crowds, in order to evaluate what these types of models can tell us about 
    crowds, which assumptions underlie them and what their strengths and 
    weaknesses are. In order to do this evaluation, we implement a computer 
    simulation of an exemplary social force model.

    In order to create this simulation, we pick an exemplary model that is 
    described in detail in the article that presents it, and analyse it in 
    detail, filling in details from other articles where necessary. Based on 
    this analysis of the model, we go from the abstract model formulation to a 
    concrete numerical simulation by filling in required details, such as  how 
    to approximate the movement of pedestrians, how to set initial conditions 
    and values, and how to implement the interaction between pedestrians and 
    walls in practice.

    From our results, it is clear that our simulation (with the right 
    parameters) exhibits reasonable pedestrian behaviour upon visual 
    inspection. While we successfully replicate some results from the 
    literature, other effects do not manifest themselves. We discuss several 
    reasons for this discrepancy, including features that are lagging from the 
    model, parameter values, effects of using random numbers to generate the 
    initial conditions and possible errors in our implementation of the model.

    Based on the results of our own simulations and our review of the social 
    force modelling field, we assess social force models and their strengths 
    and weaknesses. We conclude that social force models are not built on any 
    theories for the behaviour of crowds, but are created to replicate a set 
    of observations. As such, any confidence in their predictions must come 
    from a record of producing results fitting observations; and since the 
    field is relatively new, they have not yet reached this state. Social 
    force models do, however, provide a practical way to simulate something 
    that would otherwise be impossible to simulate. As such, they are the best 
    available way to provide e.g. guidance when designing facilities that must 
    accommodate many pedestrians, and given time the accuracy of their 
    predictions will probably increase.
\end{abstract}
\begin{abstract}
    \section*{Resume}
    \small
\end{abstract}
