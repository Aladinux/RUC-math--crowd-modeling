\section{Social force models}
\label{sec:social-forces}
One would think that the motion and dynamics of a human crowd would be
governed by complex human decision making. However, the idea of social force
models is modelling the behaviour using only a set of simple forces to
describe the behaviour of the human pedestrians that comprise a crowd.

This works by calculating a set of social forces affecting each pedestrian in the system.
These forces are not real physical forces in the sense
that they follow Newton's laws of motion, but rather a measure of the pedestrians'
motivation for acting in specific ways. However the name \emph{social forces}
is not accidental: both notation and to some extent the interpretation is
borrowed from physics.

Since social forces have so much in common with physical forces, it makes
sense to go into a bit more detail about what separates the two. If a real
force between pedestrians where to occur, the pedestrians would actually have
to touch each other. The social force works on a distance so even though
two pedestrians do not touch each other they will feel a repulsion from one another. Had it been a real force one
would feel a pressure from the force and at one point the pressure could be
so high that you could get hurt from it. This makes no sense and will of course
never happens in real life as long as people are not touching. So like the name
implies the social force is a force that comes from the social behavior of people
trying to avoid each other and should not be confused as an actual force on a pedestrian.
We will in a later chapter discus how we think this model should be interpret, when it is used to describe something real with unreal forces.

%TODO: Expand on the difference between social and physical forces.

\subsection{Variants of social force models}
In this sections we will briefly go through the forces that are used in the social force models
and mention some of the differences. This we will do to make an overview of the social force models
and try to explain some things that you could use but is not included in the model that we have
based this report on. We will not go into details with the equations at this point but just explain
the main idea about how different interactions works.

The model(\cite{self-org}) that we work with consider people to be circular when calculating the forces.
There are basically three different forces on a pedestrian $\alpha$. An attractive force from the
desired point that $\alpha$ wants to reach, a force from each of the other pedestrians and
a force from the nearest point on each of the surrounding walls and obstacles.

The force from the wall only depends on how far $\alpha$ is from the wall. The attractive force from the desired point only depends on how delayed $\alpha$ gets. Sow if $\alpha$ gets forced to move slower than he wishes he will get impatient and increase his speed and visa versa.
The force from other pedestrians has two parts, one part that depends on how the pedestrian is situated compared to the direction $\alpha$ is facing. The force will be greatest if the other pedestrian is situated directly in front of $\alpha$ and smallest if he situated directly behind $\alpha$.

In some models they also include friction between pedestrians and between pedestrians and walls. This is implemented as a tangential force. The friction between pedestrians results in clogging at doorways \cite{ABconstant}. The tangential force from the walls
means a pedestrian will tend to follow the wall when there is no or low visibility. It is described in \cite{helbing00}, and \cite{HelbingNew}.
So it is used both in older and a newer papers than the one we use.

The force from pedestrians are also changed a little so that is is not split up in two parts but the whole force is dependent
on whether another pedestrian comes from the front or from behind. This they use in \cite{ABconstant},
\cite{HelbingNew} and  \cite{helbing00}. So it is as with the tangential forces, something that has been used from time to time.

In some variations of the model they also have a more advanced force from other pedestrians.
In the advanced form the force from other pedestrians depends on the speed of the pedestrians
and also the shape of the pedestrian is made elliptical. This they use in \cite{HelbingNew} and \cite{ABconstant}.
Both are papers newer than the one we use. In \cite{ABconstant} they claim that it gives a better fit when
determining constants by making video tracking of real life situations, and when they use a speed dependent
force rather than using a force that is given by distance alone.

It should be noted that even thought there are some changes in the forces throughout the different papers,
they write in the article we use, \cite{self-org}:
\begin{quote}
 One may, of
course, take into account other details such as a velocity
dependence of the forces and non circular shaped
pedestrian bodies, but this does not have qualitative
effects on the dynamical phenomena resulting in the
simulations. In fact, most observed self-organization
phenomena are quite insensitive to the specification
of the interaction forces, while it may, of course, influence
the quantitative results.
\end{quote}

So if this is correct, we should be able to limit ourself to the model that they have in the paper we use.
Having said this we will now describe different phenomenons that the model can give in the right scenarios.


% TODO: This should contain an overview of the different social force models,
% starting with the one we picked to simulate, and explaining briefly how the
% others differ. With references for all of them.


\subsection{Results obtained from the models}
In this section we will go through different phenomenons that occurs in different scenarios and then
explain the scenarios that we have chosen to simulate. The phenomenons will be interesting since they
are easy to compare with the simulation results we get, since they require no independent calculations
but can be seen directly when running the simulations. So we will have an easy measure to see if we
are able to get our interpretation of the model to work the same ways as it is presented in the papers.
The scenarios that we have chosen to work with is based on from which scenarios, should we be able
to replicate the phenomenons we want to compare our results and the ones from \cite{self-org}.
It should be noted that all the phenomenons are mentioned in our main article \cite{self-org}.
So it should be possible to reproduce the results from the article using the model they present in the same article.

\subsubsection{Lane Formation}
Lane formation can be seen in counterflow situation, pedestrians form lanes of walking
directions, which is thought as a result of the body interaction force. When groups of pedestrians move against each other they will line up behind other
pedestrians going in the same direction as themselves and form lines throughout the corridor.
This is seen in real life situations and is one of the situations that the model are able to reproduce \cite{self-org}.
The lane formations we will try to replicate in the corridor cases.
\subsubsection{Freezing-by-Heating Effect}
This phenomenon is characterized by a blockage caused by high velocity,
that follows an increasing density. The precondition for this effect is the driving force from desired
velocity and actual velocity. The freezing by heating effect is directly connected to simulations in
physics\cite{frebyheat}. In a
normal microscopic simulation of particles, you would see particles moving more freely around as the temperature rises. A liquet being heated would at some point turn into a gas and therefore the presume would decrease.
But in the model presented by \cite{self-org} and more thoroughly explained in \cite{frebyheat}
it does not happen. When working with the social force model what you actually see is that, when the
particles/pedestrians start to move at a higher speed, they will start to blockage each other.
That is they will transit to at solid state and the density will rise and they will a some point
completely stop moving. This is found to happen in simulations with bidirectional flow, like
two groups going through a corridor in opposite directions.
This we should be able to measure when simulating the corridor case, when we raise the max desired
velocity the pedestrians should start clogging up, and we should be able to measure a high density as
a result of the increasing velocity.


\subsubsection{Oscillatory Flows at Bottlenecks}
This effect arise in situations of corridors with bidirectional flows, at bottlenecks in the corridor.
It is not possible for groups of pedestrians to pass each other at the bottleneck. Some pedestrians will have to wait for others to pass but while waiting they will get impatient and eventually push throw and then the others will have to wait. This results in a oscillatory flow.
The typical way of the oscillatory flow is that a small group of pedestrians will go through in one direction, and then a small group will go throw in the other direction.
Not so often you see single individuals passing through one at a time.
This is seen in real life experiments and therefore makes the simulations more realistic
if you can reproduce the oscillatory flow.
It should be reproducible by the model \cite{self-org}.

\subsubsection{Faster-Is-Slower Effect}
It has been observed in evacuations situations that some process takes more
time if it is performed at high speed. This is called the faster-is-slower effect. It comes
in situations like when people are trying to leave a place through a door. When people are
trying to move faster than normal they will actually start to move slower through the door.
This is according to \cite{self-org} also one of the things that the models captures. When simulations
of pedestrians leaving a room one should see that the time it takes to empty the room will take longer time
when the max desired velocity is big. In some sense the faster-is-slower effect is the same as the
freezing-by-heating effect just in a different scenario. The freezing-by-heating effect is not contained
in the bidirectional flow and people would not stop completely by the effect,
but both occurs when the desired speed is growing.
When raising the max desired velocity we should be able to see the effect, as the time it takes to
empty the room will increase as the max desired velocity is increasing.

\subsubsection{The scenarios}
The scenarios that we will work with is chosen so that we should be able to observe the four
different phenomenons that we just went through(Lane formation, freezing-by-heating, oscillatory
flows at bottlenecks and faster-is-slower).
After having gone through the phenomenons, three scenarios seems obvious to work with;
People leaving a squared room, a corridor with bidirectional flow
and a corridor with bidirectional flow and a bottleneck.
This should makes us able to see if we can recreate the results seen in the literature.

We have now explained the very basic idea of the model we use, and the variations there can be between
the social-force models. These differences is claimed to have little effect on the qualitative results.
Therefore we expect that the model we work with should have no problems replicating the different phenomenons
that we just gave a brief explanation of. After having done this, we will now continue and go into
details of the math involved in the model.

