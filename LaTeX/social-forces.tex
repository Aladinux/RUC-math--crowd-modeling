\section{Social force models}
\label{sec:social-forces}
One would think that the motion and dynamics of a human crowd would be 
governed by complex human decision making. However, the idea of social force 
models is modelling the behaviour using only a set of simple forces to 
describe the behaviour of the human pedestrians that comprise a crowd.

This works by calculating a set of social forces affecting each pedestrian, or 
agent, in the system. These forces are not real physical forces in the sense 
that they follow Newton's laws of motion, but rather a measure of the agents' 
motivation for acting in specific ways. However the name \emph{social forces} 
is not accidental: both notation and to some extent the interpretation is 
borrowed from physics. 

Since social forces have so much in common with physical forces, it makes 
sense to go into a bit more detail about what separates the two.
%TODO: Continue this

\subsection{Variants of social force models}
% TODO: Write this section
\subsection{Results obtained from the models}
% TODO: Write this section

%\subsection{Concepts for describing crowd behaviour}\label{concepts}
%In order to analyse the behaviour of crowds, we need to establish some 
%concepts to describe this behaviour. It is not obvious which concepts are 
%useful when we need to distinguish between the results we get from our 
%simulations. In this section we describe which concepts we use to describe the 
%behaviour of crowds, and why we have chosen them. This is based on the 
%literature of crowd modelling.

%One of the reasons for modelling crowds is to discover ways to make crowd 
%situations safer for pedestrians, e.g. when evacuating a building in event of 
%a fire. One of the main factors in this scenario is the \emph{efficiency} of 
%the crowd movement. This is especially important when clearing a room in the 
%event of a fire or other disaster: the faster everyone gets out, the lower is 
%the chance of someone dying from flames or smoke.

%The efficiency of the crowd it measured by \emph{how close the average speed 
%of a pedestrians is to the desired speed of the pedestrian}. This is done for 
%every pedestrian in the simulation and this gives us a measure of how unaffected the crowd moves through the environment. If every pedestrian moves at his desired speed throughout the entire simulation the efficiency will be at its maximum.

%However the measurement of efficiency alone is not sufficient to describe all 
%the aspects of the crowds behaviour. We want to be able to say something about 
%how the crowds move in specific parts of the environment for instance in a 
%narrow passage or through a door opening. For this purpose we introduce the 
%concept of \emph{flow rate}. The flow rate is a measurement of \emph{how many pedestrians that pass a point or line segment in space per unit time}.

%Furthermore we introduce the concept \emph{density}. The density is the \emph{number of people in an area of space}. The density is useful when comparing results from different simulations. 
% TODO: Why is it useful?

%The last concept we want to introduce is the \emph{geometry} of the environment by geometry we mean \emph{the shape of the room or corridor} in which the simulation is taking place. We introduce this because we are interested in examining how the shape of the environment affects the behaviour of the crowd.

%These four concepts \emph{efficiency}, \emph{flow rate}, \emph{density} and 
%\emph{geometry} will allow us to compare the results from our different simulations in some detail. For example if we change the geometry of the environment how does this change the efficiency of the crowd? If two rooms have the same geometry how does the density affect the flow rate through a door opening? If the efficiency is low does that imply a low flow rate? These are all questing that can be answered through simulations in the framework of the concepts we have just defined.    
% TODO: It is not the concepts that affect each other, it is the parameters 
% that affect the concepts.
% TODO: Move the explanation of geometry somewhere else.

%\subsection{Our case(s)}\label{ourCases}
%In this section we outline which cases we want to simulate and why.
%One of the main reasons for doing simulations is that we want to to get 
%a full understanding of the model. Doing simulations is a way to achieve this
%understanding. Secondly we want to find any limitations and weaknesses of the 
%model. Furthermore we want to simulate scenarios that have been experimentally 
%tested in \citep{self-org}.

%We start by making a series of simple simulations  where we can test isolated parts of the model and slowly increase the complexity of the simulations. Basically there are three levels of the model we want to simulate. 

%First we want to model a square room with an exit in the middle of one of the walls. In this room there will be a modest amount of people and no obstacles. The purpose of this simulation is not to compare the results with real life but rather to catch any early mistakes or misunderstandings in our implementation of the model.

%From this we move on to varying the parameters of the model too see how the changes in in the these affect the concepts of crowd behaviour we defined in section \ref{concepts}. This will be done in the same simple simulation in order for us to have a transparent look at precisely how the parameters affect the concepts. The results of these simulations will be compared with the value of the parameters presented in the literature.

%After this we will test how the model handles different geometries of the environment to see if the model starts producing unnatural crowd behaviour as the complexity of the environment increases. 

%In the end we will try so simulate more complicated situations to see if we can observe a range of self-organization phenomena that have been reported in the literature. As stated earlier there have been preformed experiments in these scenarios so we can compare our results with the results presented in the literature. These phenomena include clogging in bottlenecks, lane formation in bidirectional pedestrian flow and a faster-is-slower phenomena where some pedestrians end up moving slower than average because they are to impatient and try to overtake other pedestrians\cite{self-org}. Furthermore we want to test some phenomena  that according to the literature should make the model break down. An example of this could be to model a situation of a room filled with smoke.   

%All this should give a thorough idea about how this model should be handled and what parts of it that could be improved. or were you should be careful when using the model.
