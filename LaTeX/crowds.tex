\section{Crowds}
\label{sec:crowds}
In this section, we establish some concepts for discussing the behaviour of 
crowds in general, in order to have a basis on which we can analyse our chosen 
model. We also give an overview of the work that has been done in the field of 
crowd modelling and which approaches there has been to tackling this area, as 
well as background information on the type of model (agent-based) we are 
dealing with. Finally, we outline the cases we want to use in our simulations.

\subsection{Agent-based models and their origin}
The modelling of human behaviour using agent-based models is a 
pretty new field in modelling \cite{helbing00}. In this section, we give an 
overview of the use of agent-based models in other areas.

An agent-based model is a model that describes the behaviour of individuals 
and through simulations derive behaviour of the system being modelled.  
Agent-based models are used in many different fields.  They have their origin 
in physics where the idea of agent-based models dates as far back as 1820. 
Back then Laplace wrote \cite{simintro}:

% TODO: Language cleanup etc.

\begin{quote}
    An intelligent being who, at a given moment, knows all the forces that 
    cause nature to move and the positions of the objects that it is made 
    from, if also it is powerful enough to analyze this data, would have 
    described in the same formula the movements of the largest bodies of the 
    universe and those of the lightest atoms. Although scientific research 
    steadily approaches the abilities of this intelligent being, complete 
    prediction will always remain infinitely far away.
\end{quote}

So the main idea of agent-based models and their application has been thought 
of since long before the usage of computers. But it wasn't until the last half 
of the 20th century that it was actually possible to work with agent models 
because of the numbers of calculations needed to make use of 
them \cite{simintro}.  Since then the application of agent-based models have 
been huge and it is now used in many of different fields. 

One of the most well-known and first established (1957 \cite{MDintro}) 
agent-based model simulations is the molecular dynamics (MD)-simulations 
\cite{MDintro}. The idea of MD-models is simply to describe particles moving 
around by calculating the mechanical forces working on each of the particles.  
These kind of models allow you too see what happens with each particle and from 
that it is possible to calculate well known physical properties like 
temperature, pressure and so on . You can even see when there is a phase 
change from liquid to solid and back again. But also you can use it for much 
more complex behavior. In molecular biology and chemistry you can use 
MD-simulations to calculate the behavior of long-chain molecules \cite{MDbio}.  
In these models you even sometime add some stochastic elements to the models.  
This can be used to get some theoretical idea about how reactions happens and 
alot of other things.

But also in fields that you normally wouldn't connect with physics you see the 
application of microscopic models.
One of the more exotic fields to use microscopic models is in financial 
markets \cite{finans}.  In these kind of simulations you can describe how the 
behavior of individuals affects market dynamics.  It is of course not 
mechanical forces that controls this kind of models, but the idea of looking 
at and calculate the effects of the individual is maintained. 

Microscopic models has also been used on different types of animals like fish 
and with good results.  One of the things that have been simulated is fish 
schools \cite{fish}. In these simulations they try to the describe how a fish 
school moves around by calculating the positions of each fish and the effects 
from its nearest neighbours. In these kind of models people get some realistic  
looking behaviours that looks like what you would see in nature. Because of 
this very wide application of agent based models it seems natural to assume 
that it also could be fitted to use on human behaviour. Especially since it can 
be used on other animals, even though it could be argued that you of course 
would expect animals behavior to be more primitive than that of humans and 
therefore more easy to model.  Since microscopic models is used in a lot of 
different ways they also looks very different. That is because only the main 
idea of looking at the individual repeats throughout all of the models. So even 
though microscopic models is used very widely there is still a lot to of 
calculations and considerations that changes when the field changes and 
therefore it is still non-trivial.

\subsection{Social Force models}
%	There should be an introduction to this section.. were going to compare the 		%
%	differnt ways of simulating crowds.. but wait then it's not only relevant for 		%
%	social force models. This might fit better into the Crowds section if its rewitten	%
%	correctly.																			%

%	Some history about the social force model might serve well as an introduction here	%
In this section we will first give a short presentation on Social Force models and their history,
and then compare the features of Social Force models with the features of other models that describes
the behavior of crowds.
\\

The idea behind social force models is that each individual, pedestrian $\alpha$, in the crowd acts 
on her own from a personal aim or goal and by behavioral reactions to the environment. 
Different forces such as other pedestrians, walls and obstacles are taken into account 
when pedestrian $\alpha$ walks toward her aim or goal, so that she does not walk 
into another pedestrian or wall. The forces influencing $\alpha$'s motion 
of direction are not forces acting on $\alpha$'s body, but has to be seen as a quantity 
that describes the motivation to act. Since $\alpha$ is used to the situations 
she is normally confronted with, her reactions will be rather automatic and 
determined by her experience of which reaction will be that best. Since the reaction 
is rather automatic it is possible to put the rules of pedestrian behavior into an equation 
of motion.\cite{social-force}

\subsubsection{Comparison with other crowd models}
There are other ways to model crowds than social force models such as Rule Based models, 
Cellular Automata models and Hybrid models. Below we will make a short explanation of each of the 
other crowd models, and then compare the features of the models.\\												%

\textbf{Rule Based models}\\
Rule Based models describe the movement of pedestrians through sets of basic rules. Each 
pedestrian apply collision detection and avoidance to prevent collision with other pedestrians. 
However they do not in general apply collision response, and therefore collisions and overlappping 
of the pedestrians may occur under some circumstances. Some models have applied stopping 
rules to prevent these situations.\cite{Comparison}

\textbf{Cellular Autamata models}\\
These models are discrete, deterministic and is made up of cells like the squares in a chessboard.
An artificial intelligence approach is used in Celluar Automata (CA) defined as mathematical idealisation of physical systems in which time and space are
discrete, and physical quantities take a finite set of discrete values. Pedestrian can not collide since the floor is dicretised and the pedestrians can
only move to free adjecent cells.\cite{Comparison}

\textbf{Hybrid models}\\
The Hybrid models has been made from social force models and rule based models. 
The motion of the pedestrians is caried out like the social force models, but is also based 
on the psychological and geometrical rules. It peforms collision detection and response and rules 
are applied depending on the pedestrian's personality and the state of the environment. 
\cite{Comparison}
\\

\subsubsection{Geometrical features of the models}
Each of the different models have some geomatrical features relatively important and 
unimportant according to our simulation.

\textbf{Shaking} Wether pedestrians appear to shake when simulating the model. When 
simulating the social force models the pedestrians seems to shake when cloggings and 
queues arise. This is caused by the modification of each pedestrians postition in each 
timestep. The Hybrid, CA and Rule Based model do not appear to shake while the simulation run.

\textbf{Discrete and Continuous movement} Wether the space and time are discretised. In the CA model the pedestrians move between discritised adjecent
cells in one time step, and therefore have limited direction to go. The other models do not discretise space and therefore allow the pedestrians to move
within continuous space.
 
\textbf{Overlapping} Wether overlapping of the pedestrians is possible. The rule based models and the CA models do have collision detection but not
all have collision responce. The CA models have rules so that a pedestrian can not walk into an occupied cell, but if two pedestrians want to pass each
other they step into the cell diagonal to their own cell, and in their way cross the other pedestrian in the intersection between the four cells.
Newer models of the rule based and CA have made a stopping rule so that overlapping can not accour. The Hybrid and social force models do collision responce
to minize the risk of overlapping. \cite{Comparison}

\textbf{Pushing} Wether pedestrians can have physical contact. If the pedestrians can have physical contact they will be able to push each other in some
direction. This ability is possible when using the Hybrid and the social force model. When simulating evacuations and chaotic events it has been observed
that people do have physical contact and push each other \cite{self-org}.

\textbf{Communication} Wether the pedestrians can share information about he environment. The HiDAC and som newer rule based models allows the pedestrians
to share information about the environment and to give orders. This feature is not included in social force models and CA. \cite{Comparison}

To illustrate different features of the different models the table below sums up the above mentioned.
\begin{center}
\begin{tabular}{lllll}
 & Social Forces & Rule Based & CA & Hybrid\\
Shaking avoidance     & - & + & + & +\\
Continuous space      & + & + & - & +\\
Overlapping avoidance & + & * & - & +\\
Pushing               & + & - & - & +\\
Communication         & - & * & - & +
\end{tabular}
\end{center}
Here ``+`` indicates that the feature is possible, ``-`` indicates it is not, and ''*'' that the model has been adjusted such that the feature have
become possible. \cite{Comparison}

The features we are interested in and find relevant in our simulation are continuous space, pushing and overlapping avoidance. The continuous space is
relevant for our simulaion to be as realistic as possible since we do not want our pedestrians to have at most nine directions to go each timestep.
The feature of pushing is also important to include since in panic sitiation it has been seen that people push each. This feature will then make our
simulation more realistic.
The overlapping avoidance we find relevant since this enables pedestrians to walk through each other.

%	Finishing off with a conclusion would be neat					%


\subsection{Concepts for describing crowd behaviour}
In order to analyse the behaviour of crowds, we need to establish some 
concepts to describe this behaviour. It is not obvious which concepts  are 
useful when we need to distinguish between the results we get from our 
simulations. In this section we describe which concepts we use to describe the 
behaviour of crowds, and why we have chosen them. This is based on the 
literature of crowd modelling.

One of the reasons for modelling crowds is to discover ways to make crowd 
situations safer for pedestrians, e.g. when evacuating a building in event of 
a fire. One of the main factors in this scenario is the \emph{efficiency} of 
the crowd movement. This is especially important when clearing a room in the 
event of a fire or other disaster: the faster everyone gets out, the lower is 
the chance of someone dying from flames or smoke.

The obvious measure of efficiency is measuring how long it takes to empty a 
given room. This, however, makes the results highly dependent on the specific 
situation we are modelling (i.e. room size, number of people in the room, 
etc.). If we wish to compare results from different cases, we need a measure 
that is less dependent on the room configuration. Such a measure could be the 
\emph{flow rate}, i.e. the number of pedestrians passing a specific point (or 
line) in space per time unit.

% TODO: This section needs to be expanded with more concepts, references to 
% where we get the concepts, as well as a better arguments for why we have 
% chosen exactly these measures.

\subsection{Our case(s)}

