\section{Crowds}
\label{sec:crowds}
\subsection{Concepts for crowd behaviour}
The modelling of human behavior done by the use of agent based model, is a pretty 
new field in modelling \cite{helbing00}. Also it is a field that you normally wouldn't 
think of as being something that one would be able to predict with mathmatics. To understand 
the idea of these kinds of models completly and why some thinks that they could work, one could 
start looking at the origin of agent base model. Here we will make a short review of where 
the agent based models origins from and mention some other fields that uses semilar models. 

To start of with, lets define what an agent base model is. An agent based model is a model 
where you describe the behavior of the individual. So instead of looking at a crowd as a combined 
unit you look at the individual an try to describe them separately and let them interact with 
each other. In other words an agent based model is a special version of a microscopic model. 
Special in the way that it is made to work on humans and doesn't apply to other fields. Microscopic 
model covers in general alot of different fields. Microscopic models takes it origin in physics where 
the idea of microscopic models dates as far back as 1820. Back then Laplace wrote\cite{simintro}:

\begin{center}
\begin{minipage}[c]{5in}
\texttt{''An intelligent being who, at a given moment,
knows all the forces that cause nature to move and
the positions of the objects that it is made from, if
also it is powerful enough to analyze this data, would
have described in the same formula the movements of
the largest bodies of the universe and those of the
lightest atoms. Although scientific research steadily
approaches the abilities of this intelligent being, complete
prediction will always remain infinitely far away.''}
\end{minipage}
\end{center}

So the main idea of agent based models and their application has been thought of since long before 
the usage of computers. But it wasn't until the last half of the 20th century that it was actually 
possible to work with agent models because of the numbers of calculations needed to make use of them\cite{simintro}. 
Since then the application of agents based models have been huge and it is now used in alot of different fields. 

One of the most well-know and first established(1957\cite{MDintro}) microscopic model simulations 
is the moleculare dynamics(MD)-simulations\cite{MDintro}. The idea of MD-models is simply to 
describe particles moving around by calculating the mechanical forces working on each of the particles. 
These kind of models alow you too see what happens with each particle and from that it is posible to 
calculate well known physical properties like temperature, pressure and so on . You can even see when 
there is a phase change from liquid to solid and back again. But also you can use it for much more complex 
behavior. In molecular biology and chemistry you can use MD-simulations to calculate the behavior of long-chain molecules\cite{MDbio}. In these models you even sometime add some stochastic elements to the models. 
This can be used to get some theoretical idea about how reactions happens and alot of other things.
But also in fields that you noramlly would'nt connect with physics you see the application of microscopic models.
One of the more exotic fields to use microscopic models is in financial markets\cite{finans}. 
In these kind of simulations you can describe how the behavior of individuals affects market dynamics. 
It is of course not mechanical forces that controls this kind of models, but the idea of looking at and 
calculate the effects of the individual is maintained. 

Microscopic models has also been used on different types of animals like fish and with good results. 
One of the things that have been simmulated is fish schools\cite{fish}. In these simulations they try 
to the describe how a fish school moves around by calculating the positions of each fish and the effects 
from its nearest neighbours. In these kind of models people get some realsitc looking behaviors that 
looks like what you would see in nature. Because of this very wide application of agent based models it 
seems natural to assume that it also could be fitted to use on human behavior. Specialy since it can be 
used on other animals, even though it could be argued that you of course would expect animals behavior 
to be more primitive than that of humans and therefore more easy to model. 
Since microscopic models is used in alot of different ways they also looks very different. That is because 
only the main idea of looking at the inividual repeats throughout all of the models. So even though microscopic 
models is used very widely there is still alot to of calculations and considerations that changes when the 
field changes and therefore it is still non-trivial.

\subsection{characteristics of crowd modeling}
In this section we try to make an outline of what characteristics one could use when
describing a crowd.

%\subsection{The individual}%
%The characteristic of the individual is the following: The individual do not make 
%any complicated decisions, but is approximated to only make very basic decisions.
%
%\begin{itemize}
%\item Desired speed: each individual has a desired walking speed.\\
%\item Actual speed: The individuals speed do not always correspond to the desired speed duel to obstacles and/or other individuals blocking the desired direction.\\
%\item Maximum speed: There is a physical limit to the persons speed.\\
%\item Panic state: The person can panic, resulting in a higher speed and the person no longer respect other persons private space.\\
%\item Impatiens: Depends on the initial conditions.\\
%\item Desired direction: The individual always want to walk towards the exit.\\
%\item Influence from other individuals ($\beta$) on this individual ($\alpha$): Individuals interact and obstruct each other.\\
%\item Influence from walls and/or obsticals on the individual\\
%\item Personal sphere: The individual do'sent want other individuals ($\beta$) to invade his/hers privacy.\\
%\item Attractive forces: For example family, shopping windows etc.\\
%\end{itemize}

The characteristic for the collective group can be divided in to two subcategories: Universal 
characteristic which are the characteristics that are independent of the model used to describe
the situation. Then there are the situation/model specific characteristic. Where the situations specific 
characteristic should arise from simulations of the model, while the universal characteristic 
is used to describe the dynamics of an arbitrary group. 

\subsubsection{Universal characteristic}
A universal characteristic is a characteristic that is independent of the model.

\begin{itemize}
\item Flow rate (pedestrians/time): The number of pedestrians passing a given 
boarder during a given time.
\item Flow rate per. space: Number of pedestrians passing a given boarder during 
a given time, as a function of space. The "space" could be a doorway, with of a
 hall, specific design of a corridor etc.
\item Flow rate/average speed: The number of individuals passing a given boarder pr. 
time as a function of the average speed. (Detect at what speed clocking occurs for example)
\item Density as a function of position (x,y,$\rho$)
\end{itemize}

\subsection{Model specific characteristic}
These characteristic are characteristic which emerge from the model.

\begin{itemize}
\item Each individual is represented by two coupled differential equations representing the position and two coupled differential equations representing the velocity.
\item The initial conditions of the system is determined bye a stochastic spread.
\item At each calculation there will arise an error. The size of this error depends on the size of the time step.
\item A time step which need to be determined optimally.
\item There could be a problem determining if a deviation arise from the error or if it arise because of the stochastic spread. (What viggo said).
\item This model takes a lot of computational power. (I think)
\end{itemize}




\subsection{Our case(s)}
