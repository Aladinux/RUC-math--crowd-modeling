\section{Crowds}
\label{sec:crowds}
In this section, we establish some concepts for discussing the behaviour of 
crowds in general, in order to have a basis on which we can analyse our chosen 
model. We also give an overview of the work that has been done in the field of 
crowd modelling and which approaches there has been to tackling this area, as 
well as background information on the type of model (agent-based) we are 
dealing with. Finally, we outline the cases we want to use in our simulations.

\subsection{Concepts for describing crowd behaviour}
In order to analyse the behaviour of crowds, we need to establish some 
concepts to describe this behaviour. It is not obvious which concepts  are 
useful when we need to distinguish between the results we get from our 
simulations. In this section we describe which concepts we use to describe the 
behaviour of crowds, and why we have chosen them. This is based on the 
literature of crowd modelling.

One of the reasons for modelling crowds is to discover ways to make crowd 
situations safer for pedestrians, e.g. when evacuating a building in event of 
a fire. One of the main factors in this scenario is the \emph{efficiency} of 
the crowd movement. This is especially important when clearing a room in the 
event of a fire or other disaster: the faster everyone gets out, the lower is 
the chance of someone dying from flames or smoke.

The obvious measure of efficiency is measuring how long it takes to empty a 
given room. This, however, makes the results highly dependent on the specific 
situation we are modelling (i.e. room size, number of people in the room, 
etc.). If we wish to compare results from different cases, we need a measure 
that is less dependent on the room configuration. Such a measure could be the 
\emph{flow rate}, i.e. the number of pedestrians passing a specific point (or 
line) in space per time unit.

% TODO: This section needs to be expanded with more concepts, references to 
% where we get the concepts, as well as a better arguments for why we have 
% chosen exactly these measures.

\subsection{Concepts for crowd behaviour}
The modelling of human behaviour using agent-based models is a 
pretty new field in modelling \cite{helbing00}. In this section, we give an 
overview of the use of agent-based models in other areas.

An agent-based model is a model that describes the behaviour of individuals 
and through simulations derive behaviour of the system being modelled.  
Agent-based models are used in many different fields.  They have their origin 
in physics where the idea of agent-based models dates as far back as 1820. 
Back then Laplace wrote \cite{simintro}:

% TODO: Language cleanup etc.

\begin{quote}
    An intelligent being who, at a given moment, knows all the forces that 
    cause nature to move and the positions of the objects that it is made 
    from, if also it is powerful enough to analyze this data, would have 
    described in the same formula the movements of the largest bodies of the 
    universe and those of the lightest atoms. Although scientific research 
    steadily approaches the abilities of this intelligent being, complete 
    prediction will always remain infinitely far away.
\end{quote}

So the main idea of agent-based models and their application has been thought 
of since long before the usage of computers. But it wasn't until the last half 
of the 20th century that it was actually possible to work with agent models 
because of the numbers of calculations needed to make use of 
them \cite{simintro}.  Since then the application of agent-based models have 
been huge and it is now used in many of different fields. 

One of the most well-known and first established (1957 \cite{MDintro}) 
agent-based model simulations is the molecular dynamics (MD)-simulations 
\cite{MDintro}. The idea of MD-models is simply to describe particles moving 
around by calculating the mechanical forces working on each of the particles.  
These kind of models allow you too see what happens with each particle and from 
that it is possible to calculate well known physical properties like 
temperature, pressure and so on . You can even see when there is a phase 
change from liquid to solid and back again. But also you can use it for much 
more complex behavior. In molecular biology and chemistry you can use 
MD-simulations to calculate the behavior of long-chain molecules \cite{MDbio}.  
In these models you even sometime add some stochastic elements to the models.  
This can be used to get some theoretical idea about how reactions happens and 
alot of other things.

But also in fields that you normally wouldn't connect with physics you see the 
application of microscopic models.
One of the more exotic fields to use microscopic models is in financial 
markets \cite{finans}.  In these kind of simulations you can describe how the 
behavior of individuals affects market dynamics.  It is of course not 
mechanical forces that controls this kind of models, but the idea of looking 
at and calculate the effects of the individual is maintained. 

Microscopic models has also been used on different types of animals like fish 
and with good results.  One of the things that have been simulated is fish 
schools \cite{fish}. In these simulations they try to the describe how a fish 
school moves around by calculating the positions of each fish and the effects 
from its nearest neighbours. In these kind of models people get some realistic  
looking behaviours that looks like what you would see in nature. Because of 
this very wide application of agent based models it seems natural to assume 
that it also could be fitted to use on human behaviour. Especially since it can 
be used on other animals, even though it could be argued that you of course 
would expect animals behavior to be more primitive than that of humans and 
therefore more easy to model.  Since microscopic models is used in a lot of 
different ways they also looks very different. That is because only the main 
idea of looking at the individual repeats throughout all of the models. So even 
though microscopic models is used very widely there is still a lot to of 
calculations and considerations that changes when the field changes and 
therefore it is still non-trivial.


\subsection{Our case(s)}

