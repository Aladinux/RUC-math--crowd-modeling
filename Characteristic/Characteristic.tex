\documentclass[10pt,a4paper]{article}
\usepackage[latin1]{inputenc}
\usepackage{amsmath}
\usepackage{amsfonts}
\usepackage{amssymb}
\begin{document}
\section{Crowed model Characteristic}
The main characteristic can be divided in to two main categories, characteristic for the individual and characteristic for the collective group.
\section{The individual}
The characteristic of the individual is the following: The individual do not make any complicated decisions, but is approximated to only make very basic decisions.
\begin{itemize}
\item Desired speed: each individual has a desired walking speed.\\
\item Actual speed: The individuals speed do not always correspond to the desired speed duel to obstacles and/or other individuals blocking the desired direction.\\
\item Maximum speed: There is a physical limit to the persons speed.\\
\item Panic state: The person can panic, resulting in a higher speed and the person no longer respect other persons private space.\\
\item Impatiens: Depends on the initial conditions.\\
\item Desired direction: The individual always want to walk towards the exit.\\
\item Influence from other individuals ($\beta$) on this individual ($\alpha$): Individuals interact and obstruct each other.\\
\item Influence from walls and/or obsticals on the individual\\
\item Personal sphere: The individual do'sent want other individuals ($\beta$) to invade his/hers privacy.\\
\item Attractive forces: For example family, shopping windows etc.\\

\end{itemize}


\section{The collective group}
The characteristic for the collective group can be divided in to two subcategories: Universal characteristic and situation/model specific characteristic. Where the situations specific characteristic should arise from simulations of the model, while the universal characteristic is used to describe the dynamics of an arbitrary group. 

\subsection{Universal characteristic}
A universal characteristic is a characteristic that is independent of the model.
\begin{itemize}
\item Flow rate (pedestrians/time): The number of pedestrians passing a given boarder during a given time.
\item Flow rate per. space: Number of pedestrians passing a given boarder during a given time, as a function of space. The "space" could be a doorway, with of a hall, specific design of a corridor etc.
\item Flow rate/average speed: The number of individuals passing a given boarder pr. time as a function of the average speed. (Detect at what speed clocking occurs for example)
\item Density as a function of position (x,y,$\rho$)
\end{itemize}


\subsection{situation specific characteristic(examples)}
There will arise some special characteristic depending on a given simulation. Hopefully we will be able to identify relevant characteristic as they arise during simulations.

\begin{itemize}
\item Relative share of the flow direction $Q_1$ and $Q_2$(Special situation from the article).
\item Irregularity of time gaps at cross sections behind the intersection of two flows(special situation)
\item relative variation of time headways
\item Mean value $\bar{T_1}$ of the time headways T in flow i. (There has to be kind of a query 
\item Relative variation $\frac{\sigma_i}{\bar{T_i}}$ of time gaps at two measurement cross sections I and II(special situation)
\end{itemize}

\subsection{Model specific characteristic}
These characteristic are characteristic which emerge from the model.
\begin{itemize}
\item Each individual is represented by two coupled differential equations representing the position and two coupled differential equations representing the velocity.
\item The initial conditions of the system is determined bye a stochastic spread.
\item At each calculation there will arise an error. The size of this error depends on the size of the time step.
\item A time step which need to be determined optimally.

\item There could be a problem determining if a deviation arise from the error or if it arise because of the stochastic spread. (What viggo said).

\item This model takes a lot of computational power. (I think)

\end{itemize}



\end{document}